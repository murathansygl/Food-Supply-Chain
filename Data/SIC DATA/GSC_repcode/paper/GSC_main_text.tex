\documentclass[hidelinks,12pt,letter]{article}
\usepackage{amsmath}
\usepackage{setspace}
\onehalfspacing % \doublespacing  % 
\usepackage{amssymb}
\usepackage{amsthm}
\usepackage{natbib}
\usepackage{multicol}
\usepackage{siunitx}
\usepackage{multirow}
\usepackage[hang, flushmargin]{footmisc} 
\newdimen\footnotemargin
\footnotemargin.7em\relax
\usepackage{hyperref}
\usepackage{booktabs}
\usepackage{dcolumn}
% \usepackage{hyperref}
\usepackage{placeins}
\usepackage{threeparttable}
\usepackage{pdflscape} % landscape pages
\usepackage{afterpage} % to make text wrap around landscape pages
\usepackage{longtable} % tables that cover multiple pages
\usepackage{ulem}
\normalem % make sure \emph stays for italics
\usepackage{bigdelim}
\usepackage{color}
\usepackage[dvipsnames]{xcolor}
%\usepackage[T1]{fontenc}
%\fontsize{12}{12}
%\usepackage[T1]{fontenc}
%\usepackage[sc]{mathpazo}
\usepackage[bitstream-charter]{mathdesign}

\usepackage{tgpagella}
\usepackage[T1]{fontenc}

\usepackage[font=footnotesize]{caption} % changes figure captions
\usepackage{rotating} % for rotating documents 

\usepackage{xr}
\externaldocument{GSC_online_appendix}


% change symbols for authors
\makeatletter
\renewcommand*{\@fnsymbol}[1]{\ensuremath{\ifcase#1\or \dagger\or \ddagger\or
   \mathsection\or \mathparagraph\or \|\or **\or \dagger\dagger
   \or \ddagger\ddagger \else\@ctrerr\fi}}
\makeatother



\newcommand*{\TitleFont}{%
      \usefont{\encodingdefault}{\rmdefault}{n}{n}%
      \fontsize{19}{19}%
      \selectfont}


\newcommand{\ud}{\mathrm{d}}
\newtheorem{pred}{Prediction}
\newtheorem{hypo}{Hypothesis}
\newtheorem{defi}{Definition}

\date{}


\usepackage{graphicx}

\usepackage{fullpage}

\bibliographystyle{IO_bibstyle}

\title{\TitleFont Globalizing the Supply Chain: Firm and Industrial Support for US Trade Agreements}
\author{Iain Osgood\thanks{Assistant Professor, Department of Political Science, University of Michigan. Haven Hall, 505 S. State St, Ann Arbor MI 48104; iosgood@umich.edu. The author wishes to thank Christy Brandly, Michael Dreiling, Jeffry Frieden, Lloyd Gruber, Robert Gulotty, Noel Johnston, Andrew Kerner, Charles Lipson, Michael Plouffe, Stephen Weymouth, and Alton Worthington, as well as participants in the University of Chicago's Program on International Politics, Economics and Security and the University of Michigan's Political Economy Workshop. Supplementary material for this article is available in the Online Appendix. Replication code and data are available at the IO Data Archive (http://iojournal.org/data-archive/).}}
\begin{document}

\maketitle

\small

\begin{abstract}
\noindent 
From 1960 to 2000, manufacturing supply chains became global. To what extent has this growth in offshore outsourcing and foreign direct investment affected industrial attitudes towards trade liberalization? Using data on public positions of US firms and trade associations on all free trade agreements since 1990, it is shown that FDI and input-sourcing are the \textit{primary} drivers of support for trade liberalization. By comparison, direct import competition and export opportunities play a secondary role in shaping support for free trade agreements. This work therefore adds to the literature on the politics of globalization by providing systematic evidence of a link between global supply chains and industrial preferences, and by developing a new model of the determinants of industrial attitudes toward trade.
\bigskip
\begin{center}
\noindent \textbf{Word count}: 14159
\end{center}
\end{abstract}



\newpage 

\bigskip

\noindent 
Manufacturing supply chains grew enormously in size, speed, and complexity with the flowering of global trade after 1960. For some firms, this meant the import of intermediate inputs sourced from foreign manufacturers, or even outsourcing the manufacture of final products overseas. For an even smaller group of firms, efficiency improvements in transport, reduced trade barriers, and improving institutions in host markets permitted the movement of production facilities abroad. Though few in number, these offshore outsourcers and multinationals control an enormous quantity of manufacturing output and world trade. Beyond this elite, though, many more producers have come to benefit from imported intermediates, or through incorporation into the supply networks of export-competitive firms. The globalization of the supply chain implicates all producers, if not equally. 

How has this reorientation of supply chains outside national boundaries impacted the politics of international trade? 
Building on the literature on foreign investment and producer preferences over trade policy,\footnote{In particular, \citealt{milner1988resisting,manger2009investing}.} I argue that a complete account of industrial preferences over trade policy in the current era must place the globalization of supply networks at its center. The sourcing or production of both intermediate and final goods abroad is now a critical element of firms' production strategies. The development of a successful global supply chain often means the difference between profitability and going out of business, and trade agreements facilitate the flourishing of these networks.\footnote{\citealt{baccini2016distributional}.} Employing evidence on the public position-taking of US firms and trade associations on all US trade agreements since NAFTA, I show that opportunities to multinationalize production and source intermediate inputs are now the primary drivers of producer preferences over US trade agreements.

The predicted effects of globalization of the supply chain are large. For a typical industry, increasing the quantity of imported intermediates from its 1st quartile to its 3rd quartile in the data more than doubles the number of firms supporting trade. The probability that an industry's association supports an agreement jumps by 60\%. The effects of increased production of final products by foreign subsidiaries, a second facet of globalization of the supply chain, are nearly as great. % An increase of the same magnitude grows firm support by 53\% and the probability of association support by 39\%. 
These effects are politically important: US firms and trade associations are the primary supporters of US trade agreements and of global integration generally. Their multi-faceted campaigns on behalf of trade are extensive, sophisticated, and highly publicized. Industrial input is solicited and recorded by US trade negotiators and the Congress at all stages in the creation of trade policy, especially with international agreements. The large effects of globalization of the supply chain on industrial support therefore have large effects on the determination of US trade policy.

By contrast, the two remaining drivers of industrial preferences over trade examined here are important but secondary. Firms are vicariously integrated into the global economy through the supply of intermediates to export-competitive downstream industries located in their own home market. Where opportunities for such relationships are great, support for trade agreements is increased as the benefits of globalization propagate upwards through the domestic supply chain. Direct export opportunities and import competition also determine industrial preferences over trade, especially in industries producing homogeneous commodities, where the effects of trade are sharp and unambiguous. In industries where such impacts are muted, the globalization of the supply chain is therefore especially important as a determinant of attitudes toward trade.

The primary contribution of this paper with respect to multinational production and intermediates is empirical. Existing studies show that foreign production drives preferences in particular industries, or impacts the design of trade agreements. A similar gap -- systematic evidence across all tradables industries that globalization of the supply chain is driving preferences over trade policy -- is even more glaring in the study of intermediate inputs. The results presented below fill these voids. This paper also adds to the political science literature by developing the theory of indirect export opportunities; creating new measures of reliance on upstream imports and downstream exports; and introducing new data on public position-taking. 

Spurred by a recent set of studies on the harmful labor market consequences of trade with low wage nations,\footnote{See \citealt{david2013china,pierce2012surprisingly}.} and the resurgence of trade as a topic of electoral contestation in the 2016 presidential election, many have questioned why America rushed headlong into globalization. While strategic and diplomatic reasons are certainly important,\footnote{On the non-economic motivations behind trade agreements, see \citealt{gowa1993power, mansfield1997alliances, gowa2005exclusive, mansfield2012votes}.} public debate over US trade agreements suggests that a coalition of US firms, industries, and peak associations constitute the most important constituency in the push to liberalize trade. But what drove their demands for more globalization? This paper suggests that globalization of the supply chain and the multinationalization of production were the prime movers in generating their demands. And who will support free trade in an era of renewed hostility towards globalization? At least in the United States, the firms and industries which are most integrated into international supply chains must come to globalization's defense if an open, global economic order is to survive.

\section*{Global Supply Chains and Industry Attitudes toward Trade}
What considerations determine industry attitudes towards free trade agreements? The literature on trade politics has always focused on direct export opportunities and import competition, relegating the role of firms as vertically engaged and global consumers of inputs and final products to a secondary status. The shared fate of firms and downstream industries they supply appears to have been almost totally neglected as a systematic area of study. 

Of these three factors -- offshore production, input sourcing, and the supply of downstream industries -- multinationalization has received the most attention as a determinant of industrial preferences over trade. Case studies of particular industries or agreements have shown that multinational firms have contributed to the defense of liberal economic order and the rise of regionalism.\footnote{\citealt{milner1988trading, milner1988resisting, manger2005competition}.} A second strand of the literature has used systematic data collection across a broader array of industries to examine how trade barriers and PTA formation are shaped by multinational activity.\footnote{See \citealt{manger2014economic, manger2012vertical, lee2014outward, jensen2014influences}. \citealt{kim2016firms} 
considers trade agreement provisions to protect foreign investment; \citealt{dreiling2000class} considers multinationalization in Mexico as one determinant, among many, of public support for NAFTA. Multinationalization has received great attention as a determinant or outcome of: investment-focused international agreements in \citealt{elkins2006competing, kerner2009should, jandhyala2011three}; trade agreements, in general \citealt{buthe2008politics}; political institutions in \citealt{jensen2003democratic, jensen2012fiscal}; policy outcomes in \citealt{jensen2014unbundling}; and, partisanship in \citealt{pinto2013partisan, pinto2008politics}. These literatures focus primarily on aggregate, rather than industry-specific determinants and impacts of FDI.} This paper fills a gap between these by examining the impact of opportunities for multinationalization on the preferences of all tradables industries across a broad array of US industries for all trade agreements since 1990, precisely the era in which global supply chains have flowered.\footnote{There is also a significant literature on determinants and impact of demands for protection from foreign direct investment in host countries (or for other forms of redistribution). See, for example, \citealt{
blanchard2010reevaluating,owen2013unionization,
blanchard2015us}. Because the United States' agreement partners are generally not major home countries for FDI, this paper focuses mainly on foreign opportunities for investment for US firms, rather than threats consequent on inward FDI into the United States.} % goodman1996foreign,zeng2009foreign,

In comparison to the literature on multinationalization and trade preferences, the scale of work on sourcing intermediates abroad is small. This is especially true given the extent to which international trade consists of trade in inputs. While extant work shows that a greater scope for importing intermediates generates lobbying, campaign contributions, and lower tariffs\footnote{See \citealt{gawande2012lobbying,mccalman2004protection, gawande2000protection}.}, complementary work linking opportunities to source intermediates and industrial preferences over trade is much more scarce.\footnote{See the seminal study of industrial trade preferences, \citealt{schattschneider1935politics}. \citealt{acharya2015trade} formally examines lobbying competition between upstream and downstream industries.} The upstream impact of trade liberalization on downstream industries for input suppliers has received even less scholarly attention, though this is not for lack of substantive importance: think of the stakes for the US machine tools or auto parts industry associated with the success of the auto industry, or of the links between US agriculture and food exports. The systematic examination of the impact of export competitiveness `by proxy' is therefore a ripe target for both theoretical development and empirical investigation. Surveying the existing literature, the outstanding opportunity is to show that supply chains matter critically for determining preferences over trade policy, in our present era of global production networks more than ever. % This paper seeks to seize that opportunity. The remainder of this section motivates a renewed focus on the supply chain; develops a conceptual framework for understanding the impacts of generalized liberalization; and presents several hypotheses about the links between supply chains and support for trade among firms and industries. 

\subsubsection*{Growth in global sourcing}
Global trade has expanded enormously since the reconstruction of global economic order after World War II. Total world imports, which accounted for only 9\% of world GDP in 1950, reached 14.5\% by 1975, 21.5\% by 2000, and 27.8\% in 2011.\footnote{The data for these calculations was taken from the Penn World Tables, see \citealt{feenstra2015next}.} The growth of world trade has been accompanied by increasingly global production networks. For the largest and most successful firms, these networks comprise complex and transnational supply chains; a global division of management, marketing, and other headquarters services; and, an increasingly international orientation for sales.\footnote{\citealt{henderson2002global, unctad1999foreign}.} 

This section concentrates on the first of these aspects -- the rise of global sourcing of inputs and final products. Table \ref{globstrat} heuristically divides up the global sourcing options available to a firm which wishes to import foreign products along two dimensions. First, the firm may choose to source only inputs -- goods and services purchased by the firm to create its products -- or to source the final product itself from abroad. %\footnote{This distinction is somewhat artificial (`final products' are combined with headquarters services and so could be conceptualized as just another input) but is highly salient in the data and the identity of the producer. Firms are organized into industries based on their final product and self-identify their primary activities based on the characteristics of their final good.} 
Second, firms may choose to internalize production of the foreign-made good through a strategy of vertical foreign direct investment; to contract at arm's length with a particular foreign producer or producers; or, to purchase foreign-made goods on the open market, with no direct relationship with the foreign producer.\footnote{\citealt{antras2004global}.} %The latter strategy might be most suitable for commoditized goods, while direct contracting or vertical integration would be necessary for firm-specific inputs \citealt{antras2003firms, antras2012offshoring}.

The growth of global sourcing can be seen in part by charting the growth of traded intermediate inputs (corresponding to the first column of Table \ref{globstrat}). Intermediate inputs account for a substantial share of the enormously expanded global trade described above. One estimate holds that around 56\% of worldwide trade in goods is in intermediates.\footnote{See \citealt{miroudot2009trade}. \citealt{bergstrand2008growth} and \citealt{yeats1998just} provide lower estimates of 46\% and 30\% (as a lower bound), respectively.} The share of intermediates which are imported in the USA and elsewhere grew substantially over the 1970s and 80s, and has held steady since.\footnote{\citealt{feenstra1996globalization,campa1997evolving}. }

The growth of global sourcing is also seen in the growth of multinational production within the boundaries of the firm (corresponding to the first row of Table \ref{globstrat}). Foreign direct investment has grown enormously in relation to world trade.\footnote{\citealt{bergstrand2008growth}.} UNCTAD reports that annual flows of inward and outward FDI expanded from around 4\% of world trade flows in 1970 to a high of nearly 22\% of world trade flows in 2000, and have remained between 8 and 15\% in each year since then. Consequently, the total stock of inward FDI grew from around 6\% to nearly 30\% of world GDP from 1980 to 2011. Of course, not all FDI is oriented towards the import of inputs or final goods to the home market, as with horizontal FDI strategies that use foreign production to sell in foreign markets. As an alternative, the scale of intrafirm trade, which is more directly indicative of vertical sourcing strategies, can be seen in US data on trade with related parties. Related-party imports accounted for 46\% of all US imports in 1991, the same in 2001, and 48\% in 2011.\footnote{These figures are taken from the US Census data on Related-Party Trade, and are available at \href{www.census.gov/foreign-trade/statistics/press-release}{www.census.gov/foreign-trade/statistics/press-release}}


\setlength{\tabcolsep}{.42cm}
\begin{table}[t!]\centering
\caption{Heuristic breakdown of the choices available to a firm seeking to source goods abroad.} 
  \begin{threeparttable}
{\footnotesize \begin{tabular}{@{}p{4cm}p{3.5cm}p{3.5cm}}
\toprule
& \multicolumn{2}{c}{\uline{Stage of production}} \vspace{3pt}\\
\multicolumn{1}{@{}l}{Relationship to foreign firm} & Inputs & Final products \\ 
\midrule
Within the firm & Vertical FDI as global vertical integration & Vertical FDI as final product offshoring \\
& & \\
Outside the firm, direct & Arm's length contracting of inputs with foreign firms & Arm's length contracting of final products \\
& & \\
Outside the firm, indirect & Consumption of intermediates via wholesale or retail & Importers/middlemen\\
\bottomrule
\end{tabular}}
  \end{threeparttable}
\label{globstrat}
\end{table}

The lower right hand square of Table \ref{globstrat}, comprising wholesalers and retailers of imported products, is not examined in this paper. So only arm's length contracting of final products remains to be considered. Systematic data on this phenomenon are not publicly available; nor is there a plausible strategy available for recovering estimates on the growth of outsourcing of final products. The scale of such activities must be significant, though: think of Foxconn, which manufacturers the iPhone and iPad for Apple, and other contract manufacturers of electronics. The rest of this paper therefore only examines the foreign sourcing of final products within the boundaries of the firm while sourcing of intermediates is considered both within and beyond the boundaries of the firm.

\subsubsection*{Sourcing, sales and supply: distributive consequences of comprehensive liberalization}
Global sourcing complicates the distributive implications of broad trade liberalizations of the kind undertaken in comprehensive trade agreements.\footnote{\citealt{antras2012offshoring}.} While there are many effects arising from the increasing complexity of global production networks, building a model of supply chain politics requires some simplifying assumptions. This paper emphasizes three dimensions along which firms and industries might differ to explore the manifold impacts of trade liberalization for a given industry. 

The first of these is that trade liberalization can impact trade in own-industry final products; trade in intermediate inputs used by the industry; and trade in the products made by downstream firms who use that industry's products as inputs. For example, when evaluating a proposed trade agreement the US beef industry might consider direct exports of meat; imports of intermediates like feed or veterinary pharmaceuticals; and indirect exports via downstream industries like processed or frozen foods. %Care must be taken to distinguish among these channels, as a given firm or industry might suffer, for example, from greater direct import competition even as the downstream firms it supplies benefit from greater export sales. 

The second dimension which determines the distributive implications of trade liberalization is the firm's location. This paper considers evidence from firms that are all owned in a particular country (here, the US) but which may produce their goods either at home or abroad. These two cases may be referred to as `local producers' and `offshore producers' respectively, with the latter comprising both offshore production within the bounds of the firm (i.e. multinationalization) and offshore outsourcing to a foreign contract manufacturer. For example, a US textile producer might have supported the Dominican Republic-Central America Free Trade Agreement (CAFTA-DR) if it either owned manufacturing facilities in Central America, or outsourced production to a company there. Producers located in the US naturally hold the opposite opinion. %\footnote{The question of how US multinationals located, for example, in a third country excluded from a proposed trade agreement presents another interesting alternative which is not discussed here.} 
This reversal is general: what firms producing in the United States welcome in terms of trade flows, US firms in the same industry which offshore production will generally oppose. % The inverse of this statement is also true.

The final dimension along which firms and industries might differ is of course that some will face the trade flows identified above in greater scale then others. Some industries face severe import competition, while others are highly export-competitive, and so will benefit from liberalization. Likewise, some industries are able to deepen their global sourcing strategies and to expand their exports-by-proxy when trade agreements are signed with particular partners. Other industries lack such opportunities, and so the distributive consequences of liberalization are less sharp. 

% Collectively, we have three trade flows which might differentially impact producers in a given US industry depending on whether they are located at home or offshore, and whose effects vary based on the scale and direction of the trade flows. And of course this stylized picture can be complicated even further, for example, with the introduction of product variety, quality differentiation, relationship-specific inputs, and other features highlighted in the growing literature on global production networks.  Each of these might be individually fruitful but collectively they are overwhelming. Some factors must be emphasized and some de-emphasized to build a tractable model of producer preferences over comprehensive trade liberalization.

\subsubsection*{The proposed model and hypotheses}
Given this plethora of forces, this section proposes a model of trade politics in a world of globalizing supply chains. Some of the forces described above are treated in greater depth than others. These decisions in part reflect limitations in data, but also an informed hunch about which elements of liberalization are likely to be most impactful. The development of the hypotheses is also shaped by two empirical regularities in the data and the broader literature on corporate political activity: public industrial opposition to trade is somewhat rare, and larger firms have greater incentive and ability to lobby on trade issues than smaller ones. The hypotheses therefore concentrate on the difference between support and no support, and emphasize the preferences of the largest firms. I begin with the opportunities to globalize the supply chain that are the main focus of this paper.\medskip

\noindent \textit{Globalizing the supply chain}\; Consider an industry which faces significant import competition in its own product from some potential trade partner. A natural guess would be that this is an industry opposed to liberalization with that country. But what if that import competition is actually from domestically-owned firms with factories located in the foreign country? Such trade flows should be associated with a greater intensity of support for liberalization with that country among firms that can take advantage of new opportunities created (and existing relationships consolidated) by trade liberalization. % \footnote{Note that trade agreements (and trade liberalization generally) may serve to create new trade or consolidate and regularize existing trade patterns, as in \citealt{mansfield2008international}. Firms or industries also are likely to use existing trade patterns as a guide to the potential growth in trade. Firm or industry support for a trade agreement may therefore be rooted in retrospective evaluation of past and present gains from trade or prospective evaluation of the trade potential of a given partner. As an empirical matter, it is therefore important to acknowledge that trade flows both preceding and consequent on trade agreements may drive positiontaking on those agreements.} 
To the extent that such firms influence an industry's trade association, the association may also publicly support freer trade.

Of course, vertical multinationalization by US firms may also generate opposition to trade liberalization among those firms that continue to produce at home. And while vertical multinationalization has indeed generated overt opposition to US trade agreements, public manifestations of opposition to US trade agreements are generally scarce and so not the focus of the empirical analysis here.\footnote{See below and \citealt{osgood2017breakdown} for further discussion.} Relatedly, the firms that lobby or otherwise participate in US trade politics are generally large.\footnote{See \citealt{drope2006does,lux2011mixing,kim2017political}.} These firms have higher stakes (in absolute terms); are more politically experienced and informed; have greater resources; and may be more influential. These are the sort of firms whose support for trade is likely to be enhanced by greater opportunities for vertical FDI, rather than diminished. 

These facts push me to emphasize the positive impacts of vertical FDI on producer support for trade in the theoretical development. But because there may be distributive consequences within an industry associated with greater vertical FDI, I also consider an observable implication of intra-industry disagreement: that firm support for trade liberalization might be more robustly associated with vertical FDI than association support.  This expected difference is rooted in the idea that associations are more likely to represent the interests of a broad array of firms in an industry, comprising both multinationals and those that produce only in the home market. Because some trade associations are mainly composed of (or dominated by) larger firms, and some industries have more variation in ability to multinationalize than others, I still expect association support to increase in opportunities to multinationalize production of final goods, but less strongly.
\begin{hypo}
Industries which offshore a greater amount of production of their final products to foreign affiliates in a particular country are more likely to have support for trade liberalization with that country, especially among individual firms.
\end{hypo}

Turning from final to intermediate goods, I note that industries that rely heavily on imported intermediates have very strong stakes in trade liberalization. Sourcing intermediates at a reasonable cost can make the difference between corporate life and death for firms in an open economy.
\begin{hypo}
Industries which source a greater amount of intermediate imports from a country should be more likely to feature support for trade liberalization with that country, especially among individual firms.
\end{hypo}
\noindent As with own-industry trade flows, the import of intermediates can be disaggregated into inputs that are imported from foreign-based upstream subsidiaries and inputs imported from abroad either through contracting with foreign firms or via intermediaries. Because relatively few firms are multinationals, it seems especially likely that firm support for trade might be driven by both related and non-related party intermediate inputs while association support is driven by non-related party inputs, only. This distinction is not made in the main results because the two measures are highly correlated, however it is examined in separate models as a tentative additional finding.\footnote{The potentially negative impacts of greater exports of intermediates on firms and industries that intensively employ those intermediates is not considered although this might provide an interesting extension of the model proposed here. %I also do not pursue in depth the potential implications of relationship-intensive imports of intermediates. For example, it might be that the benefits of sourcing homogeneous commodities from abroad (e.g. gas and oil) are available to all firms whether their firm has the scale and resources to import. Prices are lower, after all. With relatively differentiated or firm-specific inputs, the gains from importing might be restricted to the largest firms with the deepest global supply chains. See \citealt{baccini2016intra, nunn2007relationship}.
}

More generally, firms may differ in their ability to take advantage of opportunities to source intermediates. Such disagreements may be less overt than the very direct competition engendered by vertical FDI, especially if intermediates are homogeneous or not relationship-specific.\footnote{\citealt{baccini2016intra}.} For the reasons described above, I do not examine here sourcing of intermediates as a driver of overt public opposition -- or diminished support -- for trade among smaller firms. I do however consider the differences between association and firm positiontaking and I expect that the latter will be more responsive to greater opportunities to import intermediates.

\medskip

\noindent \textit{The domestic supply chain and exporting-by-proxy}\; 
Some firms benefit from exporting directly, by sending their goods abroad to foreign consumers. Others benefit from export sales indirectly, by selling to local producers who then incorporate those inputs into their direct exports. This leads to support for trade liberalization `by proxy': the export-competitive firms and industries that an input-producing industry supplies are the direct beneficiaries of liberalization, but the intermediate producer's inputs end up embodied in exports, too. This argument suggests an alternative image for how supply chains might expand support for trade.
\begin{hypo}
Industries whose goods end up incorporated into downstream products which are then exported to a particular country should be more likely to have support for trade liberalization with that country.
\end{hypo}
% \noindent I leave aside a symmetric notion: that firms and industry associations might be less likely to support trade liberalization if the industries they supply are facing stiff import competition. The reason for this is that a straightforward strategy is available for determining what proportion of an industry's domestic sales end up embodied in downstream exports using input-output tables. There is no satisfying equivalent strategy for exposure to downstream import competition.

\smallskip

\noindent \textit{Direct import and export competition}\; The standard Ricardo-Viner interpretation of trade politics holds that industries should be in favor of trade if they are export-competing and opposed to liberalization if they import-competing. It is therefore essential to account for the overall competitiveness of the industry when it comes to the `ordinary trade flows' that have dominated the study of trade politics. To do so, all models described below include controls for the total exports of each industry, and the total quantity of imports not originating from US multinationals. 

A more recent literature argues that the Ricardo-Viner model is  inappropriate in industries where products are differentiated.\footnote{\citealt{osgood2016differentiated}.} 
Product differentiation is the leading explanation for intra-industry trade flows, which make an industry both import- and export-competing. Where intra-industry trade is significant, the key determinant of a firm's preferences over trade policy is whether a firm  is an exporter, not the overall trade orientation of the industry as a whole.\footnote{For survey and other evidence that firm rather than industry characteristics drive firm preferences over trade liberalization, see \citealt{plouffe2016firm}, \citealt{osgood2017charmed}, and \citealt{plouffe2012new}. For similar evidence on lobbying behaviors, see \citealt{madeira2014new}, \citealt{osgood2017breakdown}, \citealt{kim2017political} and \citealt{osgood2017industrial}.} Only an exporter has any hope of benefiting from liberalization, while non-exporters face enhanced home market competition from foreign varieties even if their home industry exports relatively more than the foreign industry.\footnote{For example, \citealt{osgood2017breakdown} 
finds that the Ricardo-Viner model is unsupported empirically in industries producing differentiated products, however it receives strong support in industries producing relatively homogeneous products, like agricultural commodities and basic minerals.} The argument that product differentiation moderates the impact of comparative advantage on support for trade is therefore re-examined here. 

\subsubsection*{The political context of firm and association positiontaking on US trade agreements}
To translate the above hypotheses into testable implications, this paper examines public expressions of support for US preferential trade agreements (PTAs) from NAFTA to the present. These agreements are valuable because they generate observable variation on the key dependent and independent variables. Firms and trade associations take pains to publicly support these agreements, but with substantial variation across industries and agreements. %No other trade-related policy issues in the United States reveals as much information about the preferences of corporate America. 
These agreements also vary enormously in their trade implications, as they span countries which differ markedly in size, endowments, comparative advantage, and the extent of FDI. PTAs have also received enormous attention from political scientists interested in international agreements, regionalism, and the evolution of the liberal order.\footnote{\citealt{dur2014design, milner1997industries, mansfield2007vetoing}. See also \citealt{mansfield2002democracies, whalley1998countries,buthe2008politics,baccini2015investment}.} To the extent that preferences over PTAs are driven by the globalization of the supply chain, this paper provides complementary evidence to work showing that opportunities to multinationalize drive PTA terms and creation.\footnote{\citealt{manger2012vertical, manger2005competition, manger2014economic}.}

% The choice of public expressions of support for PTAs also raises several questions. First, industries are not unitary actors and the component firms and associations do not always share common attitudes, which is why this paper separates the actions of firms and trade associations as described above. The findings here therefore relate to the growing literature on the organizational forms of lobbying on trade politics although that is not the main focus of the analysis.\footnote{\citealt{bombardini2012competition,madeira2016newtrade,
%kim2017political,osgood2017industrial}.}

The choice of public expressions of support for PTAs raises several questions. Most importantly, can we can draw a straight line between the underlying preferences of firms and their public expressions of support for trade agreements? Doing so requires a theory of public position-taking. This paper treats position-taking as a form of outside lobbying: a show of force by special interests directed towards government and society, and designed to convince both that there is support for a given policy.\footnote{\citealt{kollman1998outside}.} Such outside lobbying has two goals. On one hand, activating elite interests signals to politicians that there is robust support for trade liberalization, and that key interest groups accept the parameters of the agreement. On the other hand, this coalition-building activity helps to publicize the agreement -- and perhaps increase support and lobbying -- among other firms and associations that may be sitting on the sidelines. 

Under this approach, private and public preferences ought to coincide closely, if not perfectly. Firms or trade associations publicly declaim their support for a trade agreement because they genuinely support an increase in trade. The main exception to this seems to be those industries which avoid liberalization as a part of trade agreements, for example, the US sugar industry. In such cases, public support for an agreement is in affirmation of opposition to free trade. Overall, these carveouts from liberalization seem to be rare. Another concern might be that firms or associations are deterred from publicly expressing support for agreements in some systematic way that biases the revelation of genuine support. While this is a possibility, it does not seem consistent with the wide array of support across industries, across agreements, and over time.

In contrast to public support for US PTAs, public opposition to these trade agreements \textit{by firms and associations} is scarce. Some of these agreements are with relatively small trade partners: if globalization concentrates gains from trade in elite firms and spreads the costs of additional competition across many firms, there may not be a strong motive to oppose such agreements. Similarly, if intra-industry trade or variation in ability to multinationalize the supply chain divides industries, then the smaller and politically disorganized firms may be undermined in their efforts to oppose liberalization, especially if their trade association remains on the sidelines. Firms may also self-censor public expressions of opposition, for fear of being labeled `protectionists' or due to political inefficacy. Given the paucity of public expressions of opposition, this paper focuses on the greatest exploitable variation: industries where there was public support for the PTAs and industries where there was not. 

Trade agreements are certainly not unopposed, however. Labor unions, human and labor rights NGOs, environmental groups, and ordinary US voters regularly fight these trade agreements. % These actors are particularly motivated where opportunities to globalize the supply chain are significant, due to offshoring-induced job losses and corporate exploitation of weaker regulatory, labor, and environmental standards in partner countries. 
There remain bastions of protectionism in industry, too. The steel and machine tools industries featured significant public opposition to the US-Korea trade agreement, for example, and certain agricultural industries have fought to exclude their products from substantive liberalization. By comparison, the political coalition that strongly supports these agreements is uni-dimensional. Corporate America is \textit{the} special interest pushing these agreements, and its efforts absolutely dwarf the activity of any other groups that arise to support these PTAs. The public expressions of support that this paper examines are therefore not only valuable as evidence of corporate attitudes, but also politically important: there are virtually no other organized interests defending globalization.

There is every indication that the US government takes the positions of corporate America seriously, too. The Department of Commerce and Office of the United States Trade Representative solicit extensive feedback from firms and industry associations through individual submissions and Trade Advisory Committees (TACs). Firms and corporations participate in great numbers, and the activities and positions of these groups are reported to the Congress, which also holds hearings on US trade agreements which generally highlight testimony by firms and associations. Formal lobbying on US trade agreements is also extensive: hundreds of firms and associations lobby on any given agreement. As described below, public positiontaking is even larger in scale -- and highly organized among pro-trade forces -- suggesting that corporate America believes its lobbying activities are impactful.

% A final question about translating the hypotheses above into empirical models concerns how firms and industries evaluate trade agreements. I hypothesize that these evaluations are driven by the extent of trade flows among the partner countries, past, present and future. Past and present trade flows act as a guide to the comparative advantage of, and opportunities to invest in, foreign countries. Of course, such flows are limited by barriers to trade that may be reduced by trade agreements, so an approach focusing on past trade flows might fail to capture opportunities generated by reductions in barriers that are highly uneven across industries. On the other hand, trade agreements may expand trade significantly but evenly across industries, or may serve to guarantee existing trade patterns.\footnote{\citealt{mansfield2008international}.} In these cases, current trade flows are a sound guide to future trade flows, and so trade flows both current and anticipated will drive public positiontaking.\footnote{As an empirical matter, early trade flows are highly correlated with later trade flows in these data. Logged US exports, non-related party imports, and related-party imports from its trade partners in 2002-04 and 2012-14 are correlated at .83, .83, and .82, respectively. From 2012 to 2014 the correlations are .56, .64, and .70, illustrating how the year-on-year variation exceeds variation across the decades among averages.} From this perspective, trade flows are relatively stable across countries, industries and time, and agreements serve mainly to lock in and build on existing successes. Such an approach seems especially apposite for US trade agreements, all of which have been concluded with partners that are already members of the WTO, and that is the perspective adopted in this paper.

% This still leaves a variety of valid approaches to measure the importance of an agreement to a particular industry, so the one taken here is chosen for breadth, robustness and simplicity. I consider trade flows between two countries over a long span of time (2005-09 and, as robustness check for every model, 2010-14) in order to smooth over year-to-year variation in trade. These flows are used as a snapshot of the stakes involved for an industry that might secure (and give) preferential access to some country as part of a PTA. It is assumed that the impact of a PTA is proportional to the size of these stakes. A typical result in the main text therefore looks like the following: industries that sourced a greater quantity of inputs from country X from 2005 to 2009 were more likely to support a trade agreement with that country.

% This approach may be problematic for agreements that are distant from 2005-09 (particularly NAFTA) if trade flows are highly variable over time. Additional subset analyses exclude trade agreements most distant from the available measures of trade flows. I also consider alternative operationalizations which use the pre-agreement and post-agreement trade flows as alternative measures of the stakes. This approach requires cutting two agreements (NAFTA and Jordan) for the pre-agreement trade flows approach, and one agreement (NAFTA) for the post-agreement trade flows approach, because the related-party trade data is only available for 2002-2014. These measures are also noisier because trade data can be variable year-on-year, though I find very similar results to those contained in the main text. 

\subsubsection*{The political consequences of the globalization of the supply chain}
I complete the theoretical discussion by suggesting several political consequences of the globalization of the supply chain to contextualize the findings here and provide a guide for further research. Most proximately, growth in global value chains has expanded the intensity and extent of the corporate pro-trade coalition in the United States, particularly among large firms. Reciprocal trade liberalization is not just about opportunities to export: many US firms and industries already have completely free access to their most important market -- the US -- and therefore are focused on opportunities to cut costs relative to domestic and foreign competitors. Foreign sourcing is a prime way to achieve this aim.  

This growth in the weight of support for trade is likely to have three effects. First, the literature on endogenous trade policy has always emphasized the importance of special interests in driving trade policy. % \footnote{\citealt{grossman1994protection,rodrik1995political}.} 
Globalization of the supply chain reweights the preferences of producers in a pro-trade direction -- especially among large, politically active firms -- and so encourages further liberalization.\footnote{\citealt{milner1988resisting}.} Second, the policy issues of greatest interest to firms will change. Rather than pushing for either protection or market access abroad, firms with global supply chains will be concerned with market access \emph{at home}, lax rules of origin, predictable customs, protections for foreign investment, and a stable regulatory environment for offshore affiliates and partners. Finally, these strong supporters of trade may also make up for the diffident attitudes toward trade among ordinary American consumers, unpersuaded as they are by a bounty of low-priced foreign-made varieties. Special interests are at the center of trade agreement formation and passage; and firms with global supply chains are at the center of special interest activity over trade.

The rise of global production networks has also recreated fault lines over trade among producers. One way this occurs is by creating new cleavages among `pro-globalization' firms on issues that divide tradables producers and firms that both produce and consume tradables. For example, among a small subset of US industries, it is not uncommon to hear vociferous complaints over two issues: exchange rate manipulation, and unequal tax treatment due to the US's origin-based system of corporate taxation. Absent global supply chains, producers of tradable goods might be expected to agree that the dollar ought to be weaker and that the US should adopt a destination-based tax system. But firms which extensively source abroad are likely to hold the contrary positions.\footnote{\citealt{frieden2002real}, but see also \citealt{walter2008new}.} The prominence and political influence of US corporations with global supply chains helps explain the relative inaction on exchange rates, and the frosty reception towards recent proposals for a destination-based tax system.

Globalization of the supply chain has also redefined the pro-trade coalition by creating support for trade liberalization in industries that would otherwise be united in opposition to trade.\footnote{\citealt{osgood2017industrial}.} For example, US firms in an uncompetitive industry might find that their efforts at opposition are undermined by a few firms in the industry that have successfully offshored production. Because the firms that import intermediates and open foreign affiliates tend to be large, globalizing firms are especially likely to be politically influential. They have the resources and political experience to make their voice heard, and so undermine trade's opponents, creating an additional factor which pushes towards greater liberalization. % These intra-industry cleavages are reinforced by the rise of firm heterogeneity in ability to export and product differentiation, which also have the effect of making large firms support trade even in uncompetitive industries.\footnote{\citealt{osgood2016differentiated}.}

Intra-industry divisions are also likely to intersect with collective action and political institutions in important ways. On the collective action side, this paper contributes to the emergent literature arguing that trade's supporters are larger in size, fewer in number, and have greater resources than trade's opponents. Using standard collective theory, their efforts at political organization are more likely to be successful. On political institutions, the rise of large pro-trade offshorers and multinationals is likely to bias trade policy towards globalization in polities that institutionalize the influence of these actors whether through strategic cooperation between the state and industry, or through generous rules on corporate campaign contributions and lobbying. Global economic order may depend on whether the preferences and influence of global corporations are preserved. % A new focus on firm size and trade preferences, on one hand, and collective action and political institutions, on the other, is fruitful territory for further investigation.

\section*{Data and Measurement of Concepts}

\subsubsection*{Producer support for US trade agreements}
The outcome variables for this paper are built using an archive of public expressions of support by US manufacturing, mining, and agriculture firm and industry associations for all 13 US free trade agreements from NAFTA to the present.\footnote{These data were collected by the author and used in \citealt{osgood2017breakdown}, although the position-taking data on NAFTA are new to this paper.} 
These public expressions of support come in various forms. By far the largest source of data on public expressions are \emph{ad hoc} coalitions, which have formed to support every trade agreement from NAFTA to the present with the sole exception of the Jordan-US FTA. % For example, 1124 agriculture, mining, and manufacturing firms and 88 trade associations joined an \emph{ad hoc} coalition to support NAFTA. Coalitions are generally finalized and most active once negotiations are concluded, and time their efforts to coincide with Congressional debate on the trade agreement. They often create a public website; send a representative to Congressional hearings; or write letters to members of Congress or the Executive branch to express their strong support for the agreement in question.
Congressional testimony is the second main source of information on firm and associations attitudes, but I also make use of USTR submissions, government reports, and press releases. These sources require careful reading to determine if a clear and unambiguous position was taken on the trade agreement. % In the case of NAFTA, these extra sources provide information on an additional 148 firms and 36 trade associations that supported the agreement, as well as confirming 111 firm and 30 association codings from the USA*NAFTA coalition. This is typical:
Many firm and association expressions of support are repeated across multiple venues, improving confidence that public expressions are meaningful.

The top half of Table \ref{tab2} illustrates these properties of the data.\footnote{Note that two sets of agreements (the Bahrain, Morocco, and Oman agreements; and the Colombia and Panama agreements) shared a single coalition simultaneously pushing for all of the agreements (the U.S.-Middle East Free Trade Coalition and the Latin America Trade Coalition, respectively). Because these coalitions are the main source of codings for these agreements, the two sets of agreements are treated as single cases to avoid over-counting evidence of support for related agreements as support for independent agreements. All trade flow variables for these sets of agreements or agreements with multiple partners are summed across the countries.} On the left hand side, summary statistics are provided on the number of firms that supported each trade agreement in the \# Support column. For example, 1272 unique firms supported NAFTA. These codings are drawn from 20 separate sources of data recorded in the \# Sources column (3 pro-NAFTA coalitions and 17 Congressional hearings). The average number of sources for each firm, contained in the Avg. Sources column, was only 1.10 because most firms appear in a single large coalition. The analogous information on industry associations is included in the final three columns of the table.

Firm and association positions are used to construct two industry-level measures of the extent of support for a given trade agreement which are the main outcome variables in the analysis below. These variables are constructed at the 6-digit NAICS level for each individual agreement so the unit-of-analysis for this study is the industry-agreement. The data cover 403 NAICS industries in agriculture, mining and manufacturing across 10 agreement (or agreement clusters), so the total number of observations for most models is $4030$. Some of the robustness checks employ variables unobserved in particular sectors or industries, reducing the sample size.

\begin{table}[t!!]\footnotesize
\setlength{\tabcolsep}{.12cm}
\centering
\caption{Counts of US firm and association support for FTAs.}
\begin{tabular}{llcccccc}
& & \multicolumn{3}{c}{Firms} & \multicolumn{3}{c}{Associations}\\
\cmidrule(l{1em}r{1em}){3-5} \cmidrule(l{1em}r{1em}){6-8}
 Agreement & Year & \# Support & Avg. Sources & \# Sources & \# Support & Avg. Sources & \# Sources \\ 
\midrule
  % latex table generated in R 3.0.3 by xtable 1.7-1 package
% Thu Apr 06 15:43:03 2017
 NAFTA & 1994 & 1272 & 1.10 & 20 & 127 & 1.71 & 22 \\ 
  Jordan & 2001 & 7 & 1.00 & 3 & 5 & 1.00 & 2 \\ 
  AUSUS & 2004 & 135 & 1.10 & 5 & 46 & 1.25 & 11 \\ 
  Chile & 2004 & 90 & 1.00 & 4 & 32 & 1.33 & 4 \\ 
  Singapore & 2004 & 79 & 1.36 & 5 & 17 & 1.41 & 6 \\ 
  CAFTA & 2005 & 184 & 1.18 & 5 & 69 & 2.07 & 22 \\ 
  Bah/Mor/Oman & 2006 & 44 & 1.11 & 5 & 16 & 1.25 & 5 \\ 
  Peru & 2007 & 36 & 1.06 & 4 & 47 & 1.45 & 8 \\ 
  Col/Pan & 2011 & 269 & 1.12 & 14 & 121 & 1.81 & 41 \\ 
  KORUS & 2011 & 177 & 1.31 & 36 & 170 & 1.98 & 43 \\ 
  
   \bottomrule
\end{tabular}
\label{tab2}
\end{table}

The first outcome variable is a count of the number of firms that supported a given agreement in a particular industry, and is written in the regression tables below as $\#$ Firms. Note that a single firm might span multiple 6-digit NAICS industries creating dependence among the observations that is addressed below. About 48.9\% of agreement-industries have at least one firm supporting the agreement, while the average across all observations is 1.73 firms. The second outcome variable is a dichotomous measure of whether an industry trade association publicly expressed support for the trade agreement in question. A trade association publicly supported the trade agreement in about 32\% of the agreement-industry cases. 

On what basis do firms and industries evaluate trade agreements? I hypothesize that these evaluations are driven by trade flows among the partner countries, past, present and future. Past trade flows are a guide to the comparative advantage of, and opportunities to invest in, foreign countries. Of course, such flows are limited by barriers to trade that may be reduced by trade agreements, so an approach focusing on past trade flows might fail to capture opportunities generated by reductions in barriers that are highly uneven across industries. However, if trade agreements expand trade evenly across industries, or serve to guarantee existing trade patterns,\footnote{\citealt{mansfield2008international}.} current trade flows are a sound guide to future trade flows.\footnote{Earlier trade flows are highly correlated with later trade flows in these data. Logged US exports, non-related party imports, and related-party imports from its trade partners in 2002-04 and 2012-14 are correlated at .83, .83, and .82, respectively. From 2012 to 2014 the correlations are .56, .64, and .70: year-on-year variation exceeds variation across the decades among averages.} From this perspective, trade flows are relatively stable across countries, industries, and time, and agreements serve mainly to lock in and build on existing successes. Such an approach seems especially apposite for US trade agreements, all of which have been concluded with partners that are already members of the WTO. 

This still leaves a variety of valid approaches to measure the importance of an agreement to a particular industry, so the one taken here is chosen for breadth, robustness and simplicity. I consider trade flows between two countries over a long span of time (2005-09 and, as robustness check for every model, 2010-14) in order to smooth over year-to-year variation in trade. These flows are used as a snapshot of the stakes involved for an industry that might secure (and give) preferential access to some country as part of a PTA. It is assumed that the impact of a PTA is proportional to the size of these stakes. % A typical result in the main text therefore looks like the following: industries that sourced a greater quantity of inputs from country X from 2005 to 2009 were more likely to support a trade agreement with that country. 
I discuss in the robustness checks below a number of alternative approaches, including using trade prior to and after the agreement's entry-into-force.

\subsubsection*{Measurement of imports arising from US multinationals}
To measure the scale of intra-firm trade, I employ data on imports from related parties provided by the US Census Bureau at the 6-digit NAICS level.\footnote{Related-party imports are defined under the administrative provisions of the Tariff Act of 1930 \S 1401a(g) as imports entering as part of a transaction among related parties, which may include `[a]ny person directly or indirectly owning, controlling, or holding with power to vote, 5 percent or more of the outstanding voting stock or shares of any organization..."} As with all trade data employed in this paper, the related-party imports of an industry from the specific trade agreement partner are always used to explain support from that industry for that trade agreement. For example, US related-party imports in motor vehicle engine parts (NAICS 336310) from South Korea are used to explain support in that industry for the KORUS agreement.

\begin{table}[t!!]\centering\footnotesize
  \begin{threeparttable}
\setlength{\tabcolsep}{.25cm}
\caption{Summary statistics for all variables.}
\begin{tabular}{lccccl}
% \toprule
\multicolumn{1}{@{}l}{Variable} & Mean & SD & Min & Max & Industry and agreement of maximum \\
\midrule
\multicolumn{6}{@{}l}{\uline{Outcome variables}:}\vspace{1.5pt}\\ 
% latex table generated in R 3.0.3 by xtable 1.7-1 package
% Thu Apr 06 15:43:13 2017
 \# Supporting firms & 1.73 & 3.21 & 0 & 32 & 321113, Sawmills; NAFTA \\ 
  Supporting assoc. & 0.32 & 0.47 & 0 & 1 & - \\ 
   % filler
\midrule \multicolumn{6}{@{}l}{\uline{Supply chain variables}:}\vspace{2pt} \\
% latex table generated in R 3.0.3 by xtable 1.7-1 package
% Thu Apr 06 15:43:13 2017
 $\ln$ Related-party imports & 3.21 & 3.09 & 0.00 & 10.64 & 336111, Autos; NAFTA \\ 
  $\ln$ Inputs & 5.95 & 1.28 & 0.82 & 10.89 & 324110, Petroleum refineries; NAFTA \\ 
  $\ln$ Inputs (related-party) & 5.69 & 1.26 & 0.37 & 10.52 & "$\phantom{24110, Petroleum refineries; NAFT}$" \\ 
  $\ln$ Inputs (non-related-party) & 5.38 & 1.44 & 0.19 & 10.38 & "$\phantom{24110, Petroleum refineries; NAFT}$" \\ 
  $\ln$ Downstream exports & 5.53 & 2.09 & 0.00 &  9.55 & 211111, Oil and Gas Extraction; NAFTA \\ 
   % filler
\midrule 
  \multicolumn{5}{@{}l}{\uline{Own-industry trade and other controls}:}\vspace{1.5pt}  \\
 % latex table generated in R 3.0.3 by xtable 1.7-1 package
% Thu Apr 06 15:43:13 2017
 $\ln$ Imports (Non-related party) & 4.30 & 2.93 & 0.00 & 10.58 & 211111, Oil and Gas Extraction; NAFTA \\ 
  $\ln$ Exports & 5.67 & 2.60 & 0.00 & 10.18 & 336111, Autos; NAFTA \\ 
  Homogeneous & 0.10 & 0.31 & 0.00 &  1.00 & - \\ 
  Mod. differentiated & 0.28 & 0.45 & 0.00 &  1.00 & - \\ 
  Differentiated & 0.62 & 0.49 & 0.00 &  1.00 & - \\ 
  $\ln$ Sales & 9.82 & 0.50 & 8.08 & 11.74 & 324110, Petroleum refineries \\ 
   % filler
   \bottomrule
   \end{tabular}
\begin{tablenotes}[para,flushleft]
\item
\leavevmode
  \kern-\scriptspace
  \kern-\labelsep
\scriptsize{\emph{Notes}:} {The total sample is 403 industries across 10 agreements (or agreement clusters) for $N = 4030$. All trade and sales variables above are measured in logged dollars and averaged over available years from 2005-09.}
\end{tablenotes}
\label{tab3}
  \end{threeparttable}
\end{table}

While imports originating from the foreign affiliates of US corporations are related-party imports, other relationships also count as related party transactions, including familial relationships and business partnerships. These types of related-party transactions are likely to be small in scale: multinationals and large firms dominate trade flows. Foreign firms invested in the US that import from their home-market parent companies also fall under the related-party imports umbrella. This error is mitigated by the fact that the US is generally invests more in its trade agreement partners than they do in the US. Among all of the US trade agreement partners examined in this paper, US foreign direct investment in those countries exceeded their investment in the US by a factor of 2.4. For the manufacturing sector only, the number is even higher, at 2.7. I therefore employ average related-party imports from 2005 to 2009 as a proxy for imports from US multinationals. This variable is referred to as Related-party imports and is always added to one and logged. % An alternative proxy for the potential of imports to come from US multinationals is also employed in robustness checks. This variable is constructed using data on US direct investment abroad provided by the Bureau of Economic Analysis, and is described fully in the appendix. It is referred to as DIA.

\subsubsection*{Measures of imported inputs and downstream exports}
The measure of imported inputs for each 6-digit NAICS industry from each set of agreement partners is constructed using the Benchmark Input-Output table for 2002 from the Bureau of Economic Analysis. The Supplementary Direct Requirements Table reports the percentage of value added in the industry in column $i$ which is accounted for by an input from the industry in row $j$.\footnote{Note that both input and output industries are defined at the 6-digit NAICS level.} I refer to this matrix of input-output coefficients as $IO$, and $IO_{ji}$ represents the proportion of industry $i$'s value accounted for by input $j$.\footnote{Note that $\sum\limits_i IO_{ji} = 1$ ordinarily, but the diagonals of the input-output table are set to zero in order to focus on value added from inputs outside the industry itself, so as not to conflate import competition with intermediate imports.} The IO matrix is used to determine the total value of each input used by each industry, which is denoted here as $I_{ji}$. For industry $i$, the total value of input $j$ used is $I_{ji} = IO_{ji} R_i$ where $R_i$ is total industry revenue. For example, if the US auto industry is a \$100 billion a year industry and 1\% of its value comes from manufactured glass products according to the input-output table, then its total glass inputs employed are \$1 billion $= .01 \times \$100 \text{ billion}$.\footnote{These numbers are rounded to even figures to simplify the presentation, but are in the neighborhood of the true figures. The industries described here are NAICS codes 336111 (Automobile manufacturing) and  327215 (Glass product manufacturing made of purchased glass).}

The percentage of each domestically employed input which is imported from a trade partner $k$ is given by $p^{Imp}_{jk} \equiv \frac{Imp_{jk}}{R_j + Imp_j}$, where $Imp_j$ represents all imports of input $j$ and $Imp_{jk}$ represents only those imports of input $j$ coming from country $k$. Note that the denominator represents the total value of all of input $j$ used in the US. To illustrate, Canada and Mexico supply about 4\% of all glass products that US industries consume, whether made in the US or abroad. At this point, it is assumed that imported inputs are deployed proportionally across all US industries. So if a proportion $p^{Imp}_{jk}$ of input $j$ is imported from country $k$ then $p^{Imp}_{jk}I_{ji}$ represents the total amount of imports from country $k$ of input $j$ used by industry $i$. Using our running example, it would be estimated that Canada and Mexico supply $.04\times \$1 \text{ billion} = \$40$ million in glass to the US auto industry annually. 

To measure the total dependence of the US auto industry on imported inputs from Mexico and Canada, one must then sum across all of the inputs supplied to the auto industry by Mexican and Canadian industries, like glass, fabric, steel, leather, and auto parts. This yields the measure of dependence on inputs from a particular country:
\begin{defi}
The total estimated value of intermediate inputs used by industry $i$ imported from country $k$ is equal to $$ \mathrm{Inputs}_{ik} \equiv \sum\limits_j p^{Imp}_{jk}I_{ji}.$$ 
\end{defi}   
\noindent This variable is referred to in the tables below as Inputs, and is averaged across all years from 2005 to 2009 before being logged. % This measure is also decomposed into versions using only related-party imports and non-related party imports, in order to consider potential differences in inputs sourced as part of multinational supply chains within the bounds of the firm and inputs sourced abroad outside the bounds of the firm.  

The quantity of locally sold intermediate goods that end up as downstream goods which are then exported is defined analogously. The percentage of downstream (i.e. final) goods that are exported to country $k$ is $p^{Exp}_{lk} \equiv \frac{Exp_{lk}}{R_l}$ where $Exp_{lk}$ is the total exports of industry $l$ to country $k$. As above, I use the expression $I_{il}$, the total amount of input $i$ used by industry $l$. I also define $p^{Imp}_{i} \equiv \sum\limits_k p^{Imp}_{ik}$ as the percentage of a particular input that is imported from the world to subtract off imported inputs which are repackaged into downstream exports. (US input-producing industries garner no benefits from imported inputs incorporated into exports.) For example, if the US handtools industry employed \$1 billion in US-made milled steel each year as an input, and then exported $0.4\%$ of its products to South Korea annually, we would estimate that about \$4 million $=.004\times \$1 $ billion of US steel ends up in South Korea via the export of US-made handtools. To determine the total dependence of US steel mills on indirect exports to South Korea, I sum across all of the downstream industries supplied by US steel mills, like tools, electronics, transportation equipment, and machinery.

The formal definition of total US-made intermediates that are then exported as downstream goods is then given by the following:
\begin{defi}
Downstream export sales for input $i$ to country $k$ are  equal to $$\mathrm{Downstream\phantom{'}exports}_{ik} \equiv \sum\limits_l (1-p^{Imp}_i)I_{il}p^{Exp}_{lk}.$$
\end{defi}   
\noindent As with the other trade measures, this variable is  averaged across the years 2005-09 and logged.

\subsubsection*{Direct import and export sales, product differentiation, and industry size}
All of the models examined below include controls for own-industry imports and exports to the relevant agreement partner(s), to ascertain the effects of own-industry competition on support for liberalization. The export variable is simply the average exports of the industry over 2005-09 measured at the 6-digit NAICS level taken from the NAICS Related-Party Trade web application hosted by the United States Census Bureau. Related-party imports are separated from all other imports. This measure of import competition with related-party imports removed is referred to as Imports in all regression tables, leaving the parenthetical `non-related party' implied.

These trade variables both interacted with a measure of product differentiation.\footnote{\cite{rauch1999networks}.} The original measure classifies industries as exchange-traded products available on an organized commodities exchange; reference-priced products priced in industry trade publications; and all other products. This measure is concorded into the 6-digit NAICS industries employed here, and its three levels are referred to as Homogeneous, Moderately differentiated and Differentiated, respectively. Exchange-traded products share a common price and are freely substitutable, and so are homogeneous. Products that are neither exchange-traded nor reference-priced are likely to be differentiated, with each variety garnering a different price based on its product characteristics and attributes of the company producing it. All of the models include a control for industry size which is based on total sales from 2005 to 2009. Due to changes in the NAICS nomenclature, this measure is only available in 2007 for some industries. Total sales over these years are always divided by the number of years in which sales data are available. 

\subsubsection*{Empirical aims and model specification}
The hypotheses presented above suggest a set of correlations which I seek to demonstrate are present in the data. Rather than pursue bivariate relationships, each correlation is examined conditionally, holding constant other determinants of support for trade agreements. Because it is not clear theoretically which of the explanations for support for trade is causally prior, the relationships among all variables are considered simultaneously. This approach provides some confidence that a correlation between related party imports and support for trade, for example, is not spurious and in fact driven by the impact of non-related party imports on support for trade. % Modeling the explanations simultaneously also increases the amount of variance explained, improving the precision of each estimate. 
% To the extent that the correlations predicted above hold conditional on alternative explanations, the results will then support the hypothesized links between trade flows, supply chains, and firm and industrial preferences.

All of these relationships are modeled using generalized linear models. The linear predictor for the regression models is:
\begin{eqnarray*}
\theta_{ik} & = & \beta_0 + \beta_1\ln \mathrm{Rel.\phantom{'}party\phantom{'}imports}_{ik} + \beta_2\ln \mathrm{Inputs}_{ik} +\beta_3\ln\mathrm{Downstream\phantom{'}exports}_{ik} + \beta_4 \ln \mathrm{Sales}_{i}\\
&  & \phantom{\beta_0} + \boldsymbol{\beta_{5-12}}\cdot\text{Differentiation$_{i}$\phantom{'}$*$\phantom{'}($\ln$ Exports$_{ik}$ $ +$ $\ln$ Imports$_{ik}$)}. 
\end{eqnarray*} 
Note that $i$ refers to the 6-digit NAICS industry and $k$ refers to the particular agreement partner(s). The index of firm support for trade each trade agreement is a count variable which is modeled using negative binomial regression
$$ \mathrm{E}[\# \phantom{'}\mathrm{Firms}_{ik}] = \mathrm{Poisson}(\varsigma \mathrm{Exp}[{\theta}_{ik}])$$ 
where $\varsigma$ models over- or under-dispersion in the data and is distributed gamma. Association support is modeled using logistic regression:
$$ \mathrm{E}[\text{Assoc. support}_{ik}] = \mathrm{InvLogit}(\theta_{ik}). $$ 

As noted above, there is dependence among the units across industries, especially within particular agreements. In order to partially deal with these dependencies, standard errors are clustered at the agreement-3-digit NAICS level for all regression models presented. Such clustering adjusts the standard errors for idiosyncratic shocks to industries for each agreement which might raise or lower the chances of support.\footnote{The appendix explores an alternative method for addressing dependence among the outcome variables using a bootstrap method. I resample among supporting firms and associations with replacement to construct bootstrapped dependent variables and then fit the model on $5000$ bootstrapped datasets. The bootstrapped standard errors are not substantially greater than the ordinary standard errors, suggesting that multi-product firms are not contributing to gross overprecision in the estimates.} A series of random intercept models are examined in the Online Appendix towards the same end. In separate specifications, random intercepts are considered for the agreement; agreement-3-digit NAICS dummies; for each 6-digit NAICS industry; and for both agreement and 6-digit NAICS industries. I also check the same models with fixed effects to describe the relevant sources of variation in the data. 

All of the results in the text are presented as expected differences. The clustered variance-covariance matrix is used to draw from the estimated sampling distribution of the coefficients and the empirical distribution of the covariates is used for all simulations (but the variable under consideration). For example, a  reported expected difference of $.95$ on the \# Firms outcome for the Input variable indicates that increasing inputs from their 25th to their 75th percentile would predict an increase of $.95$ extra firms supporting an agreement for a typical industry. %An expected difference of $.14$ on the association variable indicates a $.14$ higher probability of an association publicly supporting agreement for the same change in the Inputs variable. 
Note that all estimates are based on the median expected difference. This mainly affects the \# Firms outcome which is skewed. Its mean is $1.73$ in the data, but the median expected number of firms is $1.12$ and so all differences should be interpreted with respect to the latter number. For example, the $.94$ estimate above implies increasing the median expected number of firms from $0.76$ (when Inputs are held at their 25th percentile) to $1.71$ (when Inputs are held at their 75th percentile). Hypothesis tests are conducted using quantiles of the simulated differences.
 
\section*{Results}
% This section mirrors the theoretical development and introduction of measures by focusing sequentially on three sets of results. First, opportunities to globalize the supply chain via own-industry FDI and the foreign sourcing of intermediates (whether inside or outside the bounds of the firm) are associated with significant increases in support for trade agreements among associations but especially firms. These predicted effects are so large that opportunities to globalize the supply chain can meaningfully be called the primary explanation for variation in attitudes toward trade among firms. Second, opportunities to export indirectly via sales of inputs to home-country exporters are associated with increases in support for trade agreements among firms, but only inconsistently among trade associations. Finally, evidence of the interaction between product differentiation and (non-related party) imports and exports is shown among associations. %Import and export flows predictably drive association positions in homogeneous good industries, but not in industries producing differentiated goods (or among firms in general).% A variety of robustness checks are also described.

\subsubsection*{Globalizing the supply chain}
Does greater potential for globalization of the supply chain through foreign direct investment or the import of foreign-made intermediates increase expected support for free trade agreements? Table \ref{tab4} show that among firms the answer to this question is unequivocally yes. The measure of the ability to multinationalize production of final products, Related-party imports, is positively associated with support for liberalization among firms. For a typical industry, increasing the logged Related-party imports measure from its 25th to its 75th percentile increases the average number of firms expected to support that trade agreement from about $.91$ to $1.38$, a difference of $.46$ firms. The third column of Table \ref{tab4} also presents a first check of this finding, by converting the trade flows into flows as a percentage of industry sales.%\footnote{These are similar to the import and export penetration variables commonly used in studies of trade. The rank of the trade flows is used due to the extreme skewness of the unlogged trade data.} %This alternative measure suggests a similarly positive link between related-party imports and support for liberalization among firms. 

\setlength{\tabcolsep}{.3cm}
\begin{table}[t!]\centering
 \caption{Predicted changes in support among firms and associations for US trade agreements.} 
  \begin{threeparttable}
{\footnotesize \begin{tabular}{lD{.}{.}{2.5}D{.}{.}{2.5}D{.}{.}{2.5}D{.}{.}{2.5}}
 & \multicolumn{2}{c}{\uline{$\ln$ variables}} & \multicolumn{2}{c}{\uline{rank \%-age vars.}} \vspace{3pt} \\
\multicolumn{1}{@{}l}{Outcome:} & \multicolumn{1}{c}{$\#$ Firms} & \multicolumn{1}{c}{Assoc.} & \multicolumn{1}{c}{$\#$ Firms} & \multicolumn{1}{c}{Assoc.}\\
\midrule
\multicolumn{5}{@{}l}{\uline{Related-party and intermediates trade}:} \vspace{2pt}\\
% latex table generated in R 3.0.3 by xtable 1.7-1 package
% Thu Apr 06 15:43:52 2017
 Rel. party imports & 0.46^{***} & 0.09^{***} & 0.22^{*} & 0.06^{*} \\ 
  Inputs & 0.95^{***} & 0.14^{***} & 0.90^{***} & 0.13^{***} \\ 
  Downstream exports & 0.13^{***} & 0.00 & 0.20^{***} & 0.00 \\ 
   \midrule \multicolumn{5}{@{}l}{\uline{Ordinary trade}:} \vspace{2pt}\\Imports $\times$ Homog. & -0.08 & -0.22^{***} & 0.00 & -0.17^{***} \\ 
  Imports $\times$ Diff. & 0.03 & 0.10^{*} & -0.32^{*} & 0.07^{*} \\ 
  Exports $\times$ Homog. & 0.15^{*} & 0.14^{***} & 0.55^{***} & 0.34^{***} \\ 
  Exports $\times$ Diff. & 0.00 & -0.03^{**} & 0.87^{***} & 0.03 \\ 
   \midrule \multicolumn{5}{@{}l}{\uline{Other controls}:} \vspace{2pt}\\Sales & 0.18^{***} & 0.01 & 0.56^{***} & 0.08^{***} \\ 
  Homog. $\rightarrow$ Mod. & 0.05 & -0.03 & 0.06 & -0.05 \\ 
  Homog. $\rightarrow$ Diff. & 0.42^{*} & -0.14 & 0.36^{*} & -0.15 \\ 
   \midrule  Pseudo-R$^2$ & 0.31 & 0.13 & 0.3 & 0.13 \\ 
  
Sample size & \multicolumn{1}{l}{\phantom{a}4030} & \multicolumn{1}{l}{\phantom{a}4030}  & \multicolumn{1}{l}{\phantom{a}4030} & \multicolumn{1}{l}{\phantom{a}4030}  \\
\bottomrule
\end{tabular}}
\begin{tablenotes}[para,flushleft]
\item
\leavevmode
  \kern-\scriptspace
  \kern-\labelsep
\scriptsize{\emph{Notes}:} {All estimates are first differences; changes in continuous variables are from 25$th$ to $75$th percentile. Median expected number of firms supporting is $1.13$; median expected probability of association support is $.29$. Standard errors are clustered at 3-digit NAICS-agreement level. \scriptsize \textsuperscript{***}$p<0.001$,\textsuperscript{**}$p<0.01$,\textsuperscript{*}$p<0.05$.}
\end{tablenotes}
  \end{threeparttable}
\label{tab4}
\end{table}

Among trade associations, the links between Related party imports and support for liberalization are also positive and substantively significant. Moving from the 25th to the 75th percentile of the Related-party imports variable increases the probability of association support by around $.09$, from $.24$ to $.33$. In the robustness checks examined below, these effects are found to be somewhat smaller or insignificant using alternative model specifications and measures, while the results on firm positiontaking and Related-party imports are quite stable. I conclude that opportunities for imports coming from foreign affiliates are a key driver of firm preferences, while the evidence is somewhat weaker that such imports drive association positiontaking. Nonetheless, association support is positively linked to multinationalization: some firms are encouraging their associations to support trade agreements that promise significant opportunities to invest abroad.

Opportunities to source intermediate inputs are even more strongly associated with support for US free trade agreements. Increasing the measure of imported intermediates coming from a particular agreement partner(s) from its 25th to its 75th percentile increases the predicted number of firms supporting the agreement in that industry by $.95$, more than doubling expected support for the agreement from $.76$ firms to $1.71$ firms. The expected difference is also large among associations, where a similar change in imported inputs increases the probability of an association supporting the agreement from $.23$ to $.37$. The predicted differences are very similar using the trade flows as a percentage of sales versions of the variables. Opportunities to import intermediates are a key driver of industrial preferences.

As described above, it is plausible that related-party intermediates and non-related party intermediates might have different impacts. Opportunities to source intermediates via direct investment in foreign affiliates would seem to be restricted to relatively few firms, and so not a likely driver of broad industrial preferences as embodied in trade associations. Table \ref{tab5} reports results from the same models as \ref{tab4}, but with the intermediate inputs variable disaggregated between related-party imports and non-related party imports (expected differences for the other variables are suppressed to conserve space). The results suggest that there is a meaningful difference between firms and associations. Firm positions are driven by opportunities to multinationalize intermediates production both within and outside the bounds of the firm. In contrast, trade association positions are not meaningfully associated with related-party imports of intermediates. % This is a similar notion to that described above on own-industry related-party imports, but is stronger and more consistent empirically. 

% The differences between firms and associations can now be succinctly summarized. The foreign production of final goods within the boundaries of the firm and the foreign sourcing of inputs made either within or outside the boundaries of the firm are all key drivers of firm support for US trade agreements. Of these three, only foreign-made inputs made outside the boundaries of the firm is a consistent driver of support for trade agreements among associations across all models examined in this paper. These patterns make sense, because multinationalization of production at any stage of the supply chain is generally only done by an elite stratum of large and highly productive firms. The sourcing of inputs made by foreign corporations is feasible for a broader set of firms, and so associations are more likely to act because a broad set of firms in their industry favors agreements with such opportunities.

\setlength{\tabcolsep}{.3cm}
\begin{table}[tbp!]\centering
\caption{Predicted changes in support among firms and associations for US trade agreements.} 
  \begin{threeparttable}
{\footnotesize \begin{tabular}{lD{.}{.}{2.5}D{.}{.}{2.5}D{.}{.}{2.5}D{.}{.}{2.5}}
% \toprule
%\midrule
 & \multicolumn{2}{c}{\uline{$\ln$ variables}} & \multicolumn{2}{c}{\uline{rank \%-age vars.}} \vspace{3pt} \\
\multicolumn{1}{@{}l}{Outcome:} & \multicolumn{1}{c}{$\#$ Firms} & \multicolumn{1}{c}{Assoc.} & \multicolumn{1}{c}{$\#$ Firms} & \multicolumn{1}{c}{Assoc.}\\
% \cmidrule(l{1em}r{1em}){2}
\midrule
\multicolumn{5}{@{}l}{\uline{Intermediates trade, related-party and not}:} \vspace{2pt}\\
 % latex table generated in R 3.0.3 by xtable 1.7-1 package
% Thu Apr 06 15:43:59 2017
 Inputs (rel. party) & 0.43^{***} & -0.03 & 0.44^{***} & -0.04^{*} \\ 
  Inputs (non-rel. party) & 0.54^{***} & 0.18^{***} & 0.57^{***} & 0.19^{***} \\ 
   \midrule  Pseudo-R$^2$ & 0.31 & 0.14 & 0.31 & 0.14 \\ 
  
Sample size & \multicolumn{1}{l}{\phantom{a}4030} & \multicolumn{1}{l}{\phantom{a}4030}  & \multicolumn{1}{l}{\phantom{a}4030} & \multicolumn{1}{l}{\phantom{a}4030}  \\
\bottomrule
\end{tabular}}
\begin{tablenotes}[para,flushleft]
\item
\leavevmode
  \kern-\scriptspace
  \kern-\labelsep
\scriptsize{\emph{Notes}:} {All remaining variables suppressed. All estimates are first differences; changes in continuous variables are from 25$th$ to $75$th percentile. Median expected number of firms supporting is $1.13$; median expected probability of association support is $.29$. Standard errors are clustered at 3-digit NAICS-agreement level. \scriptsize \textsuperscript{***}$p<0.001$,\textsuperscript{**}$p<0.01$,\textsuperscript{*}$p<0.05$.}
\end{tablenotes}
  \end{threeparttable}
\label{tab5}
\end{table}

A key argument of this paper is that the globalized supply chain has fundamentally altered the scale of industrial support for trade liberalization. To wrap up this subsection I therefore consider the extent to which the rise of global sourcing via imported intermediates and FDI have increased the rate of public expressions of support for trade liberalization among US industries. To answer this question, Table \ref{tab6} presents a series of counterfactual simulations which generate predictions about rates of support for trade liberalization under a variety of scenarios. The first scenario (called `Current levels') considers the US industrial map across all 13 agreement cases examined in this paper as they are in real life. A median expectation of 1.12 firms per industry express public support across all agreement cases; 29\% of industries have at least one association in support of a given trade agreement. 

The first counterfactual considered is where the related-party imports and total imported inputs of those industries are decreased by 90\% for every observation in the entire data. This drastic reduction is meant to mimic a world with few opportunities to globalize the supply chain. One way to interpret these estimates is the following: what type of support for liberalization would we expect to see for a typical industry if we removed almost all global sourcing opportunities while holding all other industry characteristics constant? The second way to interpret the counterfactuals is somewhat fanciful, because of concerns over partial versus general equilibrium and extrapolation, but also more vivid: What would happen to support for trade liberalization in the United States if trade agreements governed just trade in final goods, with virtually no opportunity to increase imported inputs or to multinationalize production? 

\setlength{\tabcolsep}{.12cm}
\begin{table}[tbp!]\centering
\caption{Counterfactual simulations of number of firms or association supporting trade liberalization. } 
  \begin{threeparttable}
{\footnotesize \begin{tabular}{lD{.}{.}{2.5}D{.}{.}{2.5}D{.}{.}{2.5}l}
% \toprule
\multicolumn{5}{@{}c}{\uline{Scenario: De-globalization of Supply Chains}} \vspace{3pt} \\
& \multicolumn{1}{c}{Current levels} & \multicolumn{1}{c}{Predicted levels} & \multicolumn{1}{c}{Difference} & \multicolumn{1}{c}{95\% CI}\\
\midrule
% latex table generated in R 3.0.3 by xtable 1.7-1 package
% Thu Apr 06 15:44:15 2017
 No. firms support & 1.12 & 0.63 & 0.49 & [0.42, 0.57] \\ 
  Pr. assoc. support & 0.29 & 0.20 & 0.09 & [0.07, 0.11] \\ 
   % filler
\midrule
\multicolumn{5}{@{}c}{\uline{Scenario: Sharp Deterioration in Relative Exports}} \vspace{3pt} \\
& \multicolumn{1}{c}{Current levels} & \multicolumn{1}{c}{Predicted levels} & \multicolumn{1}{c}{Difference} & \multicolumn{1}{c}{95\% CI}\\
\cmidrule{1-5}
% latex table generated in R 3.0.3 by xtable 1.7-1 package
% Thu Apr 06 15:44:32 2017
 No. firms support & 1.12 & 1.02 & 0.10 & [-0.01, 0.21] \\ 
  Pr. assoc. support & 0.29 & 0.22 & 0.07 & [0.04, 0.10] \\ 
   % filler
\midrule
\multicolumn{5}{@{}c}{\uline{Scenario: Collapse of Downstream Exports}} \vspace{3pt} \\
& \multicolumn{1}{c}{Current levels} & \multicolumn{1}{c}{Predicted levels} & \multicolumn{1}{c}{Difference} & \multicolumn{1}{c}{95\% CI}\\
\midrule
% latex table generated in R 3.0.3 by xtable 1.7-1 package
% Thu Apr 06 15:44:46 2017
 No. firms support & 1.12 & 1.06 & 0.06 & [0.02, 0.10] \\ 
  Pr. assoc. support & 0.29 & 0.29 & 0.00 & [-0.01, 0.01] \\ 
   % filler
\bottomrule
\end{tabular}}
\begin{tablenotes}[para,flushleft]
\item
\leavevmode
  \kern-\scriptspace
  \kern-\labelsep
\scriptsize{\emph{Notes}:} {All estimates are first differences from models 1 and 2 of Table \ref{tab4} among the complete sample ($N = 4030$). Changes in variables are 1/10 the observed value or 10 times the observed value. Standard errors are clustered at 3-digit NAICS-agreement level.} 
\end{tablenotes}
  \end{threeparttable}
\label{tab6}
\end{table}
Under either interpretation, the scale of the impact of intermediates and FDI on support for trade is enormous. These estimates are presented in the top third of Table \ref{tab6}. Reducing these opportunities to very low levels is predicted to reduce the number of firms supporting these agreements by 44\%; the proportion of industries with trade associations which support trade liberalization is cut by 31\%. In order to draw out the comparison with other determinants of support for liberalization, the middle third of Table \ref{tab6} considers an alternative counterfactual scenario. Instead of reducing each industry's opportunities to globalize the supply chain, I instead consider reducing their opportunities to export (to 10\% of their current totals) and increasing import inflows by a factor of 10. Despite these extreme changes, the predicted change in support for liberalization among firms is dwarfed by the changes induced by a reduction in opportunities to globalize the supply chain. Among trade associations the changes in predicted support are comparable, although they are still greater for the `de-globalizing the supply chain' scenario. Overall, the evidence overwhelmingly supports the proposition that globalizing the supply chain has grown the base of support for free trade and increased the scale of the pro-trade industrial coalition in the United States. 

% Before describing the remaining results, it is worth reflecting on these findings. The data collected for this project are comprehensive: all US preferential trade agreements since NAFTA are included, and all public expressions of support for these agreements that could be located have been included in the outcome. While the idea that globalizing the supply chain generates support for trade is not new, this is the first large-N test of this proposition using data on revealed preferences of firms and industries. The main results are robust to the inclusion of likely alternative explanations, like industry size, comparative advantage, and downstream exports. And the evidence is overwhelmingly in support of the proposition that globalizing the supply chain has grown the base of support for free trade and increased the scale of the pro-trade industrial coalition in the United States. 

% And there are several reasons to think that these expressions of support, and the concomitant growth of the pro-trade coalition, are likely to be politically meaningful. US firms and corporations account for virtually all of the special interest support for US trade agreements. Union and NGO support for these agreements, in comparison, is exceedingly sparse. US policymakers also evidently care about the positions of firms and industries. They solicit extensive feedback from corporate stakeholders through public submissions to the USTR, trade advisory committees, and congressional testimony. These efforts are significant, and are then reported in great detail in Congressional Research Service reports, for example. Corporate actors likewise devote enormous efforts to providing feedback, through the creation of public coalitions, letterwriting, lobbying, and campaign contributions. This is \textit{prima facie} evidence that such activities are politically meaningful. Because the supply chain forces examined above have such large impacts on industrial support, it is reasonable to infer  that these forces have been a key driver of US trade policy.\footnote{A more formal examination of this claim also seems like a key site for future research. For example, are members of congress more likely to vote for a trade agreement when firms in their districts have publicly supported (or lobbied) on the agreement?}

\subsubsection*{Indirect exports via supply of downstream exporters}
Table \ref{tab4} shows that Downstream exports have a statistically significant and meaningful impact on the number of firms supporting trade liberalization. Increasing the logged Downstream exports variable from its 25th to its 75th percentile increases the expected number of supporting firms from $1.10$ to $1.24$. It was hypothesized that the links between downstream exports and association support for trade agreements would be similarly positive, but the results do not support that contention. This is also true across a wide variety of robustness checks and so does not seem to be something idiosyncratic about the specifications in Table \ref{tab4}. This was not predicted.

In order to provide a sense of the impact of downstream exports, the bottom third of Table \ref{tab6} considers a reduction of all of the observations' Downstream exports by 90\% from the actual observed level. The predicted reduction in firm support for trade liberalization is statistically significant but somewhat modest compared to the other factors explored in this study. As a determinant of firms' preferences over trade agreements, it seems that indirect exports via inputs sales to downstream firms and industries that are export competitive are a consistent but second-order driver of preferences, especially in comparison with opportunities to globalize production and supply. 

\subsubsection*{`Ordinary trade': import competition and export opportunities}
The third set of results consider the impact of own-industry trade flows of the sort that are usually believed to drive trade politics. The literature has found that the impact of imports and exports is conditional on the extent of product differentiation. On the associations side, the results strongly support this claim. Where products are relatively homogeneous, greater import flows have very negative effects on the chances that a trade association supports the trade agreement. Greater export flows strongly increase support. To illustrate, increasing imports from their $25$th to their $75$th percentile generally reduces the probability a homogeneous good industry's association supports an agreement by $.22$. (Note the proportion of homogeneous good industries with public support for an agreement is $.37$.) An increase in exports increases support for an agreement by $.14$. In contrast, where products are relatively differentiated, these relationships generally fade to insignificance or even change sign. 

Among firms these relationships are generally variable in their direction across specifications. This likely reflects a lack of firm position-taking in homogeneous product industries. $59.3\%$ of homogeneous good industries have no firms at all taking a position, while $45.3\%$ of all differentiated good industries have no firms taking positions. Among industries where at least one firm took a position, that firm was the \textit{only} firm doing so among $59.4\%$ of homogeneous good industries but only $33.9\%$ differentiated good industries. More starkly, the average number of firms taking a position in homogeneous industries is just over 1, while for differentiated product industries the same figure is over 2. In other words, firm positiontaking in homogeneous good industries is scarce and idiosyncratic as a source of data, explaining the variable results.

Returning to the counterfactual claims in Table \ref{tab6}, it is worth pointing out again how impactful changes in opportunities to globalize the supply chain are compared to the ordinary trade channels which drive trade preferences in the Ricardo-Viner approach. Even the drastic proposed counterfactual (a reduction of an industries export-import ratio by a factor of 100) has only modest effects on firm positiontaking, reducing support from $1.12$ to $1.02$ firms. The effects on associations are somewhat stronger, reducing the probability of association support from $.29$ to $.22$, although they remain smaller than the effects of a major reduction in opportunities to globalize the supply chain.

\subsubsection*{Robustness of the main claims}
How robust are the main claims presented in Table \ref{tab4} to alternative measures and model specifications? % Note that the first main robustness check is reported in Table \ref{tab4} itself. The main findings which used logged total trade flows are replicated using these flows as a percentage of industry sales (and then ranked across agreement-industries to deal with skewness). A series of additional robustness checks of the main results are presented in Table \ref{tab7}. 
Columns 1 and 2 consider the same model specifications from Table \ref{tab4}, however all variables are based on the years 2010 to 2014 rather than 2005 to 2009. The main results described above are similar, especially for the variables describing opportunities to globalize the supply chain. %Note however, that the link between Related-party imports and association support is somewhat weaker, which is not altogether surprising given that associations represent a broader array of interests. 
An alternative set of specifications using trade flows two-years prior to, and two years after, agreement implementation show substantively identical results. These are provided in the Online Appendix. % These checks show that the main findings here are not driven by idiosyncratic features of trade over the period 2005-2009, but rather stable features of global trade relationships. 
Column 3 in Table \ref{tab7} uses an alternative proxy for US multinationalization constructed using data on US direct investment abroad (the variable is called DIA and is described in the online appendix).

\setlength{\tabcolsep}{.1cm}
\begin{sidewaystable}[!tbp] \centering\footnotesize
 \caption{Robustness of models from Table \ref{tab4}.} 
  \begin{threeparttable}
{\footnotesize \begin{tabular}{lD{.}{.}{2.5}D{.}{.}{2.5}D{.}{.}{2.5}D{.}{.}{2.5}D{.}{.}{2.5}D{.}{.}{2.5}D{.}{.}{2.5}D{.}{.}{2.5}D{.}{.}{2.5}D{.}{.}{2.5}}
% \toprule
%\midrule
  & \multicolumn{5}{c}{\uline{Number of Supporting Firms}} & \multicolumn{5}{c}{\uline{Association Support}}\\
  & \multicolumn{1}{c}{1} & \multicolumn{1}{c}{2} &  \multicolumn{1}{c}{3} &  \multicolumn{1}{c}{4} &  \multicolumn{1}{c}{5} &  \multicolumn{1}{c}{1} &  \multicolumn{1}{c}{2} &  \multicolumn{1}{c}{3} &  \multicolumn{1}{c}{4} &  \multicolumn{1}{c}{5} \\
  \cmidrule(l{1em}r{1em}){2-6} \cmidrule(l{1em}r{1em}){7-11}
\multicolumn{9}{@{}l}{\uline{Related-party and intermediates trade}:} \vspace{2pt}\\
 % latex table generated in R 3.0.3 by xtable 1.7-1 package
% Thu Apr 06 15:46:21 2017
 Rel. party imports & 0.35^{***} & 0.10 &  & 0.48^{***} & 0.44^{***} & 0.04 & -0.01 &  & 0.11^{***} & 0.08^{**} \\ 
  DIA &  &  & 0.47^{***} &  &  &  &  & 0.06^{***} &  &  \\ 
  Inputs & 0.87^{***} & 0.78^{***} & 0.71^{***} & 1.05^{***} & 1.05^{***} & 0.13^{***} & 0.10^{***} & 0.12^{***} & 0.11^{***} & 0.12^{***} \\ 
  Downs. exports & 0.13^{***} & 0.20^{***} & 0.12^{***} & 0.20^{***} & 0.21^{***} & 0.00 & 0.00 & 0.00 & -0.01 & -0.02^{*} \\ 
   \midrule \multicolumn{5}{@{}l}{\uline{Ordinary trade}:} \vspace{2pt}\\Imports $\times$ Homog. & -0.01 & 0.04 & 0.08 & -1.04^{***} & -0.93^{***} & -0.06^{*} & -0.03 & -0.17^{**} & -0.04 & -0.11^{*} \\ 
  Imports $\times$ Diff. & 0.16 & 0.09 & 0.34^{*} & 0.04 & 0.07 & 0.06^{**} & 0.13^{***} & 0.16^{***} & 0.07^{**} & 0.09^{***} \\ 
  Exports $\times$ Homog. & 0.26^{***} & 0.65^{***} & 0.15^{**} & 1.05^{***} & 0.94^{***} & 0.10^{***} & 0.26^{***} & 0.14^{***} & 0.12 & 0.05 \\ 
  Exports $\times$ Diff. & 0.09 & 0.84^{***} & 0.03 & 0.00 & -0.01 & 0.01 & 0.06^{*} & -0.03^{*} & -0.03^{**} & -0.03^{**} \\ 
   \midrule \multicolumn{5}{@{}l}{\uline{Other controls}:} \vspace{2pt}\\Sales & 0.17^{***} & 0.55^{***} & 0.22^{***} & 0.14^{***} & 0.17^{***} & 0.02^{*} & 0.09^{***} & 0.02^{*} & 0.02^{**} & -0.01 \\ 
  Homog. $\rightarrow$ Mod. & 0.06 & 0.09 & 0.09 & -0.03 & 0.05 & -0.04 & -0.05 & -0.02 & 0.10 & 0.08 \\ 
  Homog. $\rightarrow$ Diff. & 0.42^{*} & 0.38^{**} & 0.52^{**} & 0.24 & 0.39 & -0.14 & -0.14 & -0.12 & 0.01 & 0.11 \\ 
  Num. firms &  &  &  &  & -0.05 &  &  &  &  &  \\ 
  Assocs. budget &  &  &  &  &  &  &  &  &  & 0.06^{***} \\ 
  Num. assocs. &  &  &  &  &  &  &  &  &  & 0.19^{***} \\ 
  4-firm conc. &  &  &  &  & -0.13 &  &  &  &  & -0.07^{**} \\ 
  20-firm conc. &  &  &  &  & 0.26^{*} &  &  &  &  & 0.18^{***} \\ 
  Pct. HIIT &  &  &  &  & 0.01 &  &  &  &  & 0.00 \\ 
  Pct. VIIT &  &  &  &  & 0.10 &  &  &  &  & 0.03 \\ 
   \midrule  Pseudo-R$^2$ & 0.31 & 0.31 & 0.32 & 0.3 & 0.3 & 0.13 & 0.13 & 0.13 & 0.13 & 0.24 \\ 
  
Sample size & \multicolumn{1}{l}{\phantom{a}4030} & \multicolumn{1}{l}{\phantom{a}4030} & \multicolumn{1}{l}{\phantom{a}4030}  & \multicolumn{1}{l}{\phantom{a}3420} & \multicolumn{1}{l}{\phantom{a}3370} & \multicolumn{1}{l}{\phantom{a}4030} & \multicolumn{1}{l}{\phantom{a}4030} & \multicolumn{1}{l}{\phantom{a}4030}  & \multicolumn{1}{l}{\phantom{a}3420} & \multicolumn{1}{l}{\phantom{a}2995} \\
\bottomrule
\end{tabular}}
\begin{tablenotes}[para,flushleft]
\item
\leavevmode
  \kern-\scriptspace
  \kern-\labelsep
\scriptsize{\emph{Notes}:} {All estimates are first differences; changes in continuous variables are from 25$th$ to $75$th percentile. Median expected number of firms supporting is $1.13$; median expected probability of association support is $.29$. Standard errors are clustered at 3-digit NAICS-agreement level. \scriptsize \textsuperscript{***}$p<0.001$,\textsuperscript{**}$p<0.01$,\textsuperscript{*}$p<0.05$. Observations with missing data in models 5 and 10 are deleted.}
\end{tablenotes}
  \end{threeparttable}
\label{tab7}
\end{sidewaystable}

% Column 3 in Table \ref{tab7} uses an alternative proxy for US multinationalization constructed using data on US direct investment abroad (the variable is called DIA and is described in the online appendix). %Recall that related party imports do not perfectly track imports from US multinationals because they might include imports by foreign multinationals invested in the US. The measure is also highly correlated with non-related party imports.Using the direct investment abroad measure suggests very similar effects of opportunities to multinationalize production on support for trade agreements. It does not appear that the sources of measurement error in the Related party imports variable are unduly influencing the findings. 

Columns 4 and 5 consider only industries that produce manufactures, excluding agricultural and mining industries that might have very different patterns of public positiontaking because of systematic differences in industrial structure. Column 5 also considers an additional set of control variables (some of which are only available for manufacturing industries) which are related to lobbying on trade policy. These include: the number of firms in the industry; the number of associations in a given industry and the total budgets of those associations; the 4-firm and 20-firm concentration ratios from the US Economic Census; and the percentage of industry trade represented by horizontal and vertical intra-industry trade.\footnote{See \citealt{fontagne1997intra,manger2014economic} on this decomposition of intra-industry trade.} The main findings are entirely robust to these additional controls. The impact of vertical (that is, price-driven) intra-industry trade, an indicator of potential gains from multinationalization in North-South agreements, is strongly associated with firm support for trade agreements.\footnote{\citealt{manger2012vertical}.} % On the associations side, industry organization strongly correlates with positiontaking but does not alter the main results.

If multinationalization of sourcing and production is so important for public positiontaking, a natural question is whether these forces also drive formal lobbying by firms and associations. This question is examined in the online appendix using analogous outcomes (the number of firms that lobbied and a dummy variable for whether an association lobbied.) I find that the main results of this paper carry over to the case of formal lobbying. Related-party imports, Inputs, and Downstream exports are positively associated with firm lobbying on US trade agreements, though the results are strongest for the Inputs variable. 

\section*{Conclusion: The Changed Nature of the Pro-Trade Coalition}
Opportunities to source intermediates and multinationalize production are the most important drivers of industrial preferences over trade agreements. Import competition and export opportunities, whether direct or indirect, play important but secondary roles. What impact has this elevation of the global supply chain as the main motive of support for globalization had on American trade politics? Rather than focus on the role of intermediates and FDI as drivers of the formation of trade agreements or in stabilizing a liberal international order, points that are well covered in the literature discussed above, I instead focus on an intermediate outcome: the changing size and shape of the pro-trade coalition in the United States. Each of the three consequences for the nature of America's pro-trade coalition described below have the effect of pushing US trade policy toward greater openness and so are proximate explanations for the broader trade policy outcomes emphasized in the extant literature.

Growth of multinational supply chains has created \textit{more support} for trade liberalization than would otherwise exist. The growth of intermediates sourcing from abroad has made US tradeables producers into consumers, too. But unlike the ordinary American consumer, who also benefits from trade, these firms and their associations are politically organized and highly influential.\footnote{\citealt{gilens2014testing}.} They publicly support trade, they form coalitions to push their pro-trade agenda, and they lobby members of Congress to pass agreements. The growth in multinationalization of production has had a similar effect, creating new opportunities for American business which expand the weight of interest groups pushing for free trade. And while the growth of multinationals must also create new opposition to trade among uncompetitive US firms producing in the US, the political advantage lies with pro-trade MNCs owing to their size and political resources. 

Global supply chains have also changed the nature of US trade politics by \textit{creating support for trade in uncompetitive industries}. In the Ricardo-Viner model of trade preferences commonly used to describe early post-war American trade politics, comparative advantage industries benefit from trade while comparative disadvantage industries are harmed by it. We therefore do not expect to see support for trade liberalization in industries at a comparative disadvantage. Do opportunities to source abroad fundamentally change this dynamic? This amounts to a question of whether there are differential effects of opportunities for FDI and importing intermediates across the competitiveness spectrum. If these opportunities increase support for liberalization in net-exporting industries only, we might say that the intensity of support for trade has changed but not its nature.  

To examine this question, I replicate the main models from Table \ref{tab4} by dividing up industries according to their tercile in the distribution of the export-import ratio.\footnote{These models are presented in Appendix A.} These tests show that the nature of trade politics has indeed changed. Related party imports and intermediate inputs are associated with big increases in support for liberalization in uncompetitive industries as well competitive or neutral industries. So even as these forces have super-charged support for trade in net-exporting industries, they have undermined opposition to trade in net-importing industries by creating a vocal constituency in favor of liberalization. This is true among both firms and trade associations. For example, increasing inputs from their 25th to their 75th percentile is predicted to increase the probability an association supports an agreement from $.16$ to $.36$ in a typical net-importing industry. This undermining of opposition is also abetted by the rise of product variety, which turns the largest firms in relatively uncompetitive industries into supporters of trade liberalization, perfectly able to find a home for their varieties in export markets. These forces combine to sharply undermine efforts to oppose liberalization in net-importing industries. 

Finally, the rise of global supply chains has changed the focus of trade politicking \textit{from associations to firms}. In earlier eras of American trade politics, industries were represented politically by their associations. In the present era, firms are just as likely to strike out on their own. Foreign direct investment is a natural explanation for such a pattern, but so is product differentiation which turns the largest firms into trade winners even in uncompetitive industries. Sourcing intermediates can have similar effects if those inputs are relationship-specific, and only the largest firms are able to contract directly with foreign input suppliers. Each of these forces has the same outcome: only a subset of firms in any given industry can benefit from these opportunities. While one consequence of this is that industries might see internal divisions, another is that the largest firms see highly concentrated gains from trade liberalization. And those outsized gains also coincide with financial resources for lobbying efforts, experience in dealing with politicians, and any other political advantage associated with scale. These large US firms, many of them multinationals, stand as the vanguard of the pro-trade coalition in the United States. And they have succeeded in their aim of knitting the US economy ever more tightly into the global marketplace. 

% Moving beyond preferences and organization to outcomes, it is possible to connect these new forces with the changed scope and content of the global trade agenda. Growth in globalization of the supply chain ought to lead to deeper trade liberalization, as the stakes are raised for trade liberalization and the size and resources of the pro-trade coalition expands. The proliferation of preferential trade agreements testifies to this trend, even if the Doha Round of WTO talks has run aground, due mainly to the agricultural sector (and its thin supply chains and limited multinationalization). Globalization of the supply chain also ought to reinforce itself through the expanded breadth of preferential trade agreements. Investment chapters and dispute settlement provisions are a natural site for multinationals to press for further institutional development. Firms sourcing offshore and multinationals alike will demand measures to facilitate trade, through the reduction of customs barriers; the harmonization of corporate, environmental, SPS, and labor regulation; and increases in regulatory and customs transparency. These expanded trade agenda items move well beyond the market access issues of yesterday (reductions in tariffs and NTBs) and even of today (TBTs, e-commerce, trade in services). Today's ``global factories'' thrive where trade flows freely in all directions, and where regulation is harmonized to create a globalized common market.\footnote{See \citealt{buckley2009impact} for the origins of this term.} Whether these actors can see this agenda through given renewed strains on globalization -- inequality, technological change, growing multipolarity -- remains to be seen.


\small
\clearpage
\bibliography{WAIDbib}


\begin{center}
\textbf{SUPPORTING INFORMATION}
\end{center}
The following additional materials are available in the online appendices:\\
\textbf{Appendix A}: Additional Models.\\
\textbf{Appendix B}: Cases and Data.\\


\end{document}




