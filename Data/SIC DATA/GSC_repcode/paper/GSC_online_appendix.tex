\documentclass[hidelinks,12pt,letter]{article}
\usepackage{amsmath}
\usepackage{setspace}
\onehalfspacing % \doublespacing  % 
\usepackage{amssymb}
\usepackage{amsthm}
\usepackage{natbib}
\usepackage{multicol}
\usepackage{siunitx}
\usepackage{multirow}
\usepackage[hang, flushmargin]{footmisc} 
\newdimen\footnotemargin
\footnotemargin.7em\relax
\usepackage{hyperref}
\usepackage{booktabs}
\usepackage{dcolumn}
% \usepackage{hyperref}
\usepackage{placeins}
\usepackage{threeparttable}
\usepackage{pdflscape} % landscape pages
\usepackage{afterpage} % to make text wrap around landscape pages
\usepackage{longtable} % tables that cover multiple pages
\usepackage{ulem}
\normalem % make sure \emph stays for italics
\usepackage{bigdelim}
\usepackage{color}
\usepackage[dvipsnames]{xcolor}
%\usepackage[T1]{fontenc}
%\fontsize{12}{12}
%\usepackage[T1]{fontenc}
%\usepackage[sc]{mathpazo}
\usepackage[bitstream-charter]{mathdesign}

\usepackage{tgpagella}
\usepackage[T1]{fontenc}

\usepackage[font=footnotesize]{caption} % changes figure captions
\usepackage{rotating} % for rotating documents 

\usepackage{xr}
\externaldocument{GSC_main_text}


% change symbols for authors
\makeatletter
\renewcommand*{\@fnsymbol}[1]{\ensuremath{\ifcase#1\or \dagger\or \ddagger\or
   \mathsection\or \mathparagraph\or \|\or **\or \dagger\dagger
   \or \ddagger\ddagger \else\@ctrerr\fi}}
\makeatother



\newcommand*{\TitleFont}{%
      \usefont{\encodingdefault}{\rmdefault}{n}{n}%
      \fontsize{19}{19}%
      \selectfont}


\newcommand{\ud}{\mathrm{d}}
\newtheorem{pred}{Prediction}
\newtheorem{hypo}{Hypothesis}
\newtheorem{defi}{Definition}

\date{}


\usepackage{graphicx}

\usepackage{fullpage}

\bibliographystyle{apsr}

\title{\TitleFont \textbf{Online Appendix}: Globalizing the Supply Chain: Firm and Industrial Support for US Trade Agreements}
\author{Iain Osgood\thanks{Assistant Professor, Department of Political Science, University of Michigan. Haven Hall, 505 S. State St, Ann Arbor MI 48104; iosgood@umich.edu.}}
\begin{document}

\maketitle

\small




\newpage 

\bigskip



\FloatBarrier

\newpage \footnotesize
\setcounter{table}{0}
\renewcommand{\thetable}{A\arabic{table}}
\setcounter{figure}{0}
\renewcommand{\thefigure}{A\arabic{figure}}



\newpage


\begin{center}
\textbf{\uppercase{\huge{For Online Publication Only}}}
\end{center}
\section*{Appendix A: Additional Models}

\FloatBarrier

\subsubsection*{Extending the Approach to Lobbying}
The results presented above show that the globalization of both sales and supply is the primary determinant of public position taking by firms and trade associations on US free trade agreements. If that is the case, then it stands to reason that formal lobbying of the Congress and Executive agencies ought to be similarly driven by these forces. Holding ordinary trade competition constant, for example, we ought to see greater lobbying where opportunities to multinationalize production are present. Likewise, incentives to lobby to privately express support or influence the terms or implementation of an agreement might be greater where intermediate inputs coming from the agreement are significant or the chances to export via downstream proxies are large. 

The results of the lobbying models presented in Table \ref{lobreg} support each of these contentions, particularly among firms. To interpret these results, first note that the median number of firms expected to lobby per industry-agreement is $.55$. (NAFTA is excluded from this analysis because it preceded the Lobbying Disclosure Act.) Increasing related party imports from their 25th to their 75th percentile increases the expected number of firms lobbying on an agreement by $.09$, while increasing inputs sourced from that country increases expected firm support by $.35$ firms. Downstream exports are also positively associated with increases in firm lobbying.

The links between association lobbying and globalizing the supply chain are weaker overall: only sourcing intermediates is significantly associated with lobbying. These results are consistent with the findings on public positiontaking and may again reflect conflicts within industries given that opportunities to multinationalize production are generally restricted to a minority of firms.  

Not all firms which lobby are necessarily supporters of trade agreements, so an immediate question is whether these results hold among lobbying firms and associations that were publicly identified as supporters of these agreements. The final two columns of Table \ref{lobreg} show that they mainly do, with the exception of downstream exports for firms. (These results use the logged total versions of explanatory variables.) Note that an average of $.23$ supporting firms lobby for each industry for a given agreement. Finally, Table \ref{lobbyinrob} in the Online Appendix recreates the main robustness checks presented for position taking in Table \ref{tab7} but using the lobbying based outcomes (among all firms and associations). As with the position-taking outcomes, I find that the main patterns established are robust to alternative measures and controls, and among the subsample of manufacturing industries only.

\begin{table}[t!!]\footnotesize
\setlength{\tabcolsep}{.12cm}
\centering
\caption{Counts of US association and firm lobbying on 13 FTAs.}
\begin{tabular}{llcccccc}
\toprule
%\midrule
   \multicolumn{8}{@{}l}{\textbf{Lobbying Activity}:} \\
& & \multicolumn{3}{c}{Firms} & \multicolumn{3}{c}{Associations}\\
\cmidrule(l{1em}r{1em}){3-5} \cmidrule(l{1em}r{1em}){6-8}
 Agreement & Year & Total & Support & No/Opp. & Total & Support & No/Opp. \\ 
\midrule
NAFTA & 1994 & - & - & - & - & - & - \\ 
 Jordan & 2001 & 1 & 0 & 1 & 2 & 0 & 2 \\ 
  AUSFTA & 2004 & 22 & 15 & 7 & 8 & 4 & 4 \\ 
  Chile & 2004 & 7 & 7 & 0 & 1 & 0 & 1 \\ 
  Singapore & 2004 & 7 & 4 & 3 & 2 & 0 & 2 \\ 
  CAFTA-DR & 2005 & 67 & 22 & 45 & 26 & 15 & 11 \\ 
  Bah/Mor/Omn & 2006 & 31 & 14 & 17 & 18 & 2 & 16 \\ 
  Peru & 2007 & 58 & 12 & 46 & 28 & 15 & 13 \\ 
  Pan/Col & 2011 & 132 & 56 & 76 & 59 & 40 & 19 \\ 
  KORUS & 2011 & 150 & 63 & 87 & 74 & 59 & 15 \\   
   \bottomrule
\end{tabular}
\label{lobcodings}
\end{table}

\setlength{\tabcolsep}{.1cm}
\begin{table}[t!]\centering
\caption{Predicted changes in lobbying among firms and associations on US trade agreements.} 
  \begin{threeparttable}
{\footnotesize \begin{tabular}{lD{.}{.}{2.5}D{.}{.}{2.5}D{.}{.}{2.5}D{.}{.}{2.5}D{.}{.}{2.5}D{.}{.}{2.5}}
\toprule
%\midrule
 & \multicolumn{2}{c}{\uline{$\ln$ variables}} & \multicolumn{2}{c}{\uline{rank \%-age vars.}} & \multicolumn{2}{c}{\uline{supporters only}} \vspace{3pt} \\
\multicolumn{1}{@{}l}{\textbf{Lobbying outcomes}:} & \multicolumn{1}{c}{$\#$ Firms} & \multicolumn{1}{c}{Assoc.} & \multicolumn{1}{c}{$\#$ Firms} & \multicolumn{1}{c}{Assoc.} & \multicolumn{1}{c}{$\#$ Firms} & \multicolumn{1}{c}{Assoc.}\\
% \cmidrule(l{1em}r{1em}){2}
\midrule
\multicolumn{5}{@{}l}{\uline{Related-party and intermediates trade}:} \vspace{2pt}\\
% latex table generated in R 3.0.3 by xtable 1.7-1 package
% Thu Apr 06 15:47:06 2017
 Rel. party imports & 0.09 & 0.02 & 0.03 & 0.00 & 0.06^{*} & 0.02 \\ 
  Inputs & 0.35^{***} & 0.05^{***} & 0.25^{***} & 0.02^{*} & 0.14^{***} & 0.06^{***} \\ 
  Downstream exports & 0.05^{**} & 0.01 & 0.05 & 0.00 & 0.00 & 0.00 \\ 
   \midrule \multicolumn{5}{@{}l}{\uline{Ordinary trade}:} \vspace{2pt}\\Imports $\times$ Homog. & 0.27^{**} & -0.04 & 0.33^{**} & -0.01 & 0.09 & -0.13^{**} \\ 
  Imports $\times$ Diff. & 0.16 & 0.12^{***} & 0.00 & 0.11^{***} & 0.03 & 0.11^{***} \\ 
  Exports $\times$ Homog. & 0.02 & 0.05^{*} & 0.21^{*} & 0.12^{***} & 0.02 & 0.07^{***} \\ 
  Exports $\times$ Diff. & -0.02 & -0.03^{***} & 0.29^{***} & -0.01 & 0.00 & -0.03^{***} \\ 
   \midrule \multicolumn{5}{@{}l}{\uline{Other controls}:} \vspace{2pt}\\Sales & 0.18^{***} & 0.01 & 0.34^{***} & 0.04^{***} & 0.06^{***} & -0.01 \\ 
  Homog. $\rightarrow$ Mod. & -0.04 & -0.02 & -0.02 & -0.03 & -0.06 & 0.02 \\ 
  Homog. $\rightarrow$ Diff. & 0.16 & -0.05 & 0.17 & -0.05 & 0.00 & -0.01 \\ 
   \midrule  Pseudo-R$^2$ & 0.14 & 0.04 & 0.13 & 0.04 & 0.08 & 0.06 \\ 
  
Sample size & \multicolumn{1}{l}{\phantom{a}3636} & \multicolumn{1}{l}{\phantom{a}3636}  & \multicolumn{1}{l}{\phantom{a}3636} & \multicolumn{1}{l}{\phantom{a}3636}  & \multicolumn{1}{l}{\phantom{a}3636} & \multicolumn{1}{l}{\phantom{a}3636}  \\
\bottomrule
\end{tabular}}
\begin{tablenotes}[para,flushleft]
\item
\leavevmode
  \kern-\scriptspace
  \kern-\labelsep
\scriptsize{\emph{Notes}:} {All estimates are first differences; changes in continuous variables are from 25$th$ to $75$th percentile. Median expected number of firms lobbying is $.55$; median expected probability of association lobbying is $.14$. Standard errors are clustered at 3-digit NAICS-agreement level. \scriptsize \textsuperscript{***}$p<0.001$,\textsuperscript{**}$p<0.01$,\textsuperscript{*}$p<0.05$.}
\end{tablenotes}
  \end{threeparttable}
\label{lobreg}
\end{table}

These results on lobbying corroborate the findings above on positiontaking but on a separate independent variable that represents another key facet of trade politics. Opportunities to source intermediate inputs from trade agreement partners are strongly associated with the decision by both firms and associations to lobby the US government on trade agreements. Related party imports and downstream exports are also linked to the decision to lobby, although mainly among firms. Given our results on positiontaking above, the most plausible interpretation of this is that greater existing and potential trade with agreement partners leads firms and associations to push the agreements with those partners in fora both public and private.


\setlength{\tabcolsep}{.1cm}
\begin{sidewaystable}[!tbp] \centering
\footnotesize
 \caption{Robustness of models from Table \ref{lobreg}.} 
  \begin{threeparttable}
{\footnotesize \begin{tabular}{lD{.}{.}{2.5}D{.}{.}{2.5}D{.}{.}{2.5}D{.}{.}{2.5}D{.}{.}{2.5}D{.}{.}{2.5}D{.}{.}{2.5}D{.}{.}{2.5}D{.}{.}{2.5}D{.}{.}{2.5}}
\toprule
%\midrule
  & \multicolumn{5}{c}{\uline{Number of Supporting Firms}} & \multicolumn{5}{c}{\uline{Association Support}}\\
  & \multicolumn{1}{c}{1} & \multicolumn{1}{c}{2} &  \multicolumn{1}{c}{3} &  \multicolumn{1}{c}{4} &  \multicolumn{1}{c}{5} &  \multicolumn{1}{c}{1} &  \multicolumn{1}{c}{2} &  \multicolumn{1}{c}{3} &  \multicolumn{1}{c}{4} &  \multicolumn{1}{c}{5} \\
  \cmidrule(l{1em}r{1em}){2-6} \cmidrule(l{1em}r{1em}){7-11}
\multicolumn{9}{@{}l}{\uline{Related-party and intermediates trade}:} \vspace{2pt}\\
% latex table generated in R 3.0.3 by xtable 1.7-1 package
% Thu Apr 06 15:48:17 2017
 Rel. party imports & 0.13^{*} & 0.07 &  & 0.10 & 0.08 & 0.02 & 0.00 &  & 0.03 & 0.02 \\ 
  DIA &  &  & 0.01 &  &  &  &  & -0.03^{***} &  &  \\ 
  Inputs & 0.37^{***} & 0.27^{***} & 0.36^{***} & 0.36^{***} & 0.33^{***} & 0.06^{***} & 0.03^{**} & 0.06^{***} & 0.03^{***} & 0.03^{***} \\ 
  Downs. exports & 0.05^{**} & 0.05 & 0.06^{***} & 0.09^{**} & 0.10^{***} & 0.00 & 0.00 & 0.01 & 0.01 & 0.00 \\ 
   \midrule \multicolumn{5}{@{}l}{\uline{Ordinary trade}:} \vspace{2pt}\\Imports $\times$ Homog. & 0.18 & 0.26^{**} & 0.33^{***} & -0.02 & -0.01 & -0.03 & 0.00 & -0.02 & 0.10^{*} & 0.02^{*} \\ 
  Imports $\times$ Diff. & 0.06 & 0.04 & 0.28^{**} & 0.14 & 0.20^{*} & 0.10^{***} & 0.12^{***} & 0.14^{***} & 0.09^{***} & 0.05^{*} \\ 
  Exports $\times$ Homog. & 0.12^{*} & 0.32^{***} & 0.02 & 0.05 & 0.02 & 0.03 & 0.09^{**} & 0.05^{*} & -0.08^{**} & -0.04^{***} \\ 
  Exports $\times$ Diff. & 0.04 & 0.32^{***} & -0.02 & -0.04 & -0.07^{*} & -0.02^{**} & 0.00 & -0.04^{***} & -0.03^{***} & -0.02^{*} \\ 
   \midrule \multicolumn{5}{@{}l}{\uline{Other controls}:} \vspace{2pt}\\Sales & 0.12^{***} & 0.31^{***} & 0.18^{***} & 0.18^{***} & 0.21^{***} & 0.00 & 0.04^{***} & 0.01 & 0.01 & -0.01 \\ 
  Homog. $\rightarrow$ Mod. & -0.05 & -0.02 & -0.03 & -0.22 & -0.02 & -0.02 & -0.03 & -0.02 & 0.07 & 0.04 \\ 
  Homog. $\rightarrow$ Diff. & 0.14 & 0.15 & 0.17 & -0.01 & 0.20 & -0.05 & -0.05 & -0.05 & 0.04 & 0.06 \\ 
  Num. firms &  &  &  &  & -0.06 &  &  &  &  &  \\ 
  Assocs. budget &  &  &  &  &  &  &  &  &  & 0.01^{*} \\ 
  Num. assocs. &  &  &  &  &  &  &  &  &  & 0.13^{***} \\ 
  4-firm conc. &  &  &  &  & -0.30^{***} &  &  &  &  & -0.04^{*} \\ 
  20-firm conc. &  &  &  &  & 0.60^{***} &  &  &  &  & 0.07^{**} \\ 
  Pct. HIIT &  &  &  &  & -0.02^{**} &  &  &  &  & 0.00 \\ 
  Pct. VIIT &  &  &  &  & 0.11^{*} &  &  &  &  & 0.04^{**} \\ 
   \midrule  Pseudo-R$^2$ & 0.15 & 0.15 & 0.14 & 0.12 & 0.15 & 0.05 & 0.05 & 0.05 & 0.04 & 0.13 \\ 
  
Sample size & \multicolumn{1}{l}{\phantom{a}3636} & \multicolumn{1}{l}{\phantom{a}3636} & \multicolumn{1}{l}{\phantom{a}3636} & \multicolumn{1}{l}{\phantom{a}3078} & \multicolumn{1}{l}{\phantom{a}3033} & \multicolumn{1}{l}{\phantom{a}3636} & \multicolumn{1}{l}{\phantom{a}3636} & \multicolumn{1}{l}{\phantom{a}3636} & \multicolumn{1}{l}{\phantom{a}3078} & \multicolumn{1}{l}{\phantom{a}2697} \\
\bottomrule
\end{tabular}}
\begin{tablenotes}[para,flushleft]
\item
\leavevmode
  \kern-\scriptspace
  \kern-\labelsep
\scriptsize{\emph{Notes}:} {All estimates are first differences; changes in continuous variables are from 25$th$ to $75$th percentile. \scriptsize \textsuperscript{***}$p<0.001$,\textsuperscript{**}$p<0.01$,\textsuperscript{*}$p<0.05$.}
\end{tablenotes}
  \end{threeparttable}
 
\label{lobbyinrob}
\end{sidewaystable}


\FloatBarrier
\newpage
\subsubsection*{Bootstrapped standard errors}
One of the questions raised in the main text concerns multi-product firms (and associations). A single expression of support by one firm or association can lead to multiple 6-digit NAICS industries having an extra firm or association support the agreement. Naturally, then, there is dependence among the units analyzed, violating the usual assumption of independent observations. The solution pursued in the main text -- to cluster standard errors at the 3-digit NAICS-agreement level -- ameliorates some of this problem but imperfectly, because most firms don't span an entire 3-digit NAICS code. 

As an alternative, I explored using a two-stage bootstrap procedure. In the first stage, I resample from the set of all supporting firms and associations for each agreement. For example, I randomly sample (with replacement) 177 firms from the 177 total firms that supported the KORUS agreement. This bootstrap sample of supporting firms is then mapped into the outcome variable $\#$ Firms in the expected way. In total, I created 5000 bootstrap samples of the firm and association outcomes variables which are used to resolve the issues associated with multi-product firms and associations. The extra variance that is likely to be generated by multi-product firms is then incorporated into the estimation.

The second stage of the procedure then does an ordinary bootstrap over all of the observations (including the outcome variables). Because I have 5000 bootstrapped outcome variables, I do the same number of bootstrap samples for the complete data. The estimates from these models are then presented in Table \ref{bs} which examines only the total versions of the explanatory variables not the rank \%-age versions. These models are presented simply as regression coefficients, because the focus here is on evaluating the relative difference in the standard errors. (Note that due to sampling variability there are slight differences in the coefficient estimates for firms. Because the association variable is dichotomous, and so can only switch to zero in the bootstrapped samples, the coefficients are noticeably different.) Models 1 and 3 represent the OLS estimates of the standard errors; models 2 and 4 are my bootstrapped standard errors. 

Examining the differences between the two, it is evident that the boostrapped standard errors are not significantly different from the OLS standard errors. In general, it appears that any additional variance created by multi-product firms is not sufficient to grossly alter the uncertainty around the coefficient estimates for the model. I infer from this that simulated first differences in the main text will not be significantly different either. I therefore retain the clustered standard errors in the main text.

\setlength{\tabcolsep}{.12cm}
\begin{table}[h!]\centering
 \caption{Bootstrapped standard errors for model 1 from Table \ref{tab4}.} 
  \begin{threeparttable}
{\footnotesize \begin{tabular}{lD{.}{.}{2.5}D{.}{.}{2.5}D{.}{.}{2.5}D{.}{.}{2.5}}
\toprule
% \midrule
 & \multicolumn{2}{c}{\# Firms} & \multicolumn{2}{c}{Assoc. support} \\
 & \multicolumn{1}{c}{1}  & \multicolumn{1}{c}{2} & \multicolumn{1}{c}{3} & \multicolumn{1}{c}{4} \\ 
\midrule
\multicolumn{5}{@{}l}{\uline{Related-party and intermediates trade}:} \vspace{2pt}\\
% latex table generated in R 3.0.3 by xtable 1.7-1 package
% Thu Apr 06 15:49:17 2017
 Rel. party imports &  0.022 &  0.022 &   0.869 &   0.612 \\ 
   & 0.003^{***} & 0.003^{***} &  0.183^{***} &  0.208^{**} \\ 
  Inputs (rel. party) &  0.091 &  0.086 &   3.306 &   2.470 \\ 
   & 0.004^{***} & 0.006^{***} &  0.305^{***} &  0.379^{***} \\ 
  Downstream exports &  0.012 &  0.012 &   0.088 &   0.109 \\ 
   & 0.002^{***} & 0.003^{***} &  0.155 &  0.200 \\ 
   \midrule \multicolumn{5}{@{}l}{\uline{Ordinary trade}:} \vspace{2pt}\\Imports (non. rel. party) & -0.242 & -0.234 &  -5.443 &  -3.800 \\ 
   & 0.079^{**} & 0.070^{***} &  5.433 &  5.083 \\ 
  Exports & -0.099 & -0.104 &   1.454 &   2.151 \\ 
   & 0.078 & 0.076 &  5.364 &  5.292 \\ 
  Imports $\times$ Mod. diff. &  0.002 &  0.002 &   1.574 &   1.085 \\ 
   & 0.006 & 0.007 &  0.418^{***} &  0.534^{*} \\ 
  Imports $\times$ Diff. &  0.006 &  0.006 &   2.077 &   1.630 \\ 
   & 0.006 & 0.007 &  0.398^{***} &  0.505^{**} \\ 
  Exports $\times$ Mod. diff. &  0.013 &  0.014 &  -1.132 &  -0.859 \\ 
   & 0.007 & 0.008 &  0.495^{*} &  0.524 \\ 
  Exports $\times$ Diff. &  0.010 &  0.010 &  -2.797 &  -2.311 \\ 
   & 0.007 & 0.008 &  0.480^{***} &  0.513^{***} \\ 
   \midrule \multicolumn{5}{@{}l}{\uline{Other controls}:} \vspace{2pt}\\Sales & -0.006 & -0.006 &  -1.567 &  -1.213 \\ 
   & 0.005 & 0.006 &  0.353^{***} &  0.478^{*} \\ 
  Mod. diff & -0.004 & -0.004 &   2.237 &   1.856 \\ 
   & 0.006 & 0.006 &  0.410^{***} &  0.450^{***} \\ 
  Diff. &  0.064 &  0.061 &   1.227 &   0.938 \\ 
   & 0.010^{***} & 0.013^{***} &  0.681 &  0.948 \\ 
  Intercept & -2.305 & -2.226 & -51.691 & -41.433 \\ 
   & 0.206^{***} & 0.253^{***} & 14.190^{***} & 19.459^{*} \\ 
  
% \midrule 
\bottomrule
\end{tabular}}
\begin{tablenotes}[para,flushleft]
\item
\leavevmode
  \kern-\scriptspace
  \kern-\labelsep
\scriptsize{\emph{Notes}:} {\scriptsize \textsuperscript{***}$p<0.001$,\textsuperscript{**}$p<0.01$,\textsuperscript{*}$p<0.05$. Bootstrapped standard errors contained in columns 2 and 4 as described in text.}
\end{tablenotes}
  \end{threeparttable}
\label{bs}
\end{table}

\FloatBarrier
\newpage
\subsubsection*{Alternative model specifications}
This section considers a series of alternative specifications of the main model in order to explore whether the main results are owing to a misspecification of the main model. Recall that the main results in the paper use trade values averaged over the years 2005 to 2009 for all of the main trade flow variables. Such an approach has several merits and several flaws. On the positive side, all of the variables are measured over the time span (the related-party imports data starts in 2002 and ends in 2014). Using a five-year period also permits smoothing over year-to-year variation in annual US-partner country trade flows, which can vary significantly from year to year. This also ensures comparability on the measure from the perspective of the independent variables. For example, if trade falls across the board in 2009, we won't be mistakenly attributing outsized support for the Peru trade agreement to that fact. 

Of course, this comparability comes at a cost from the perspective of the dependent variables. For example, a reader of this paper noted that the temporal disjuncture between support for NAFTA in 1994 and the trade patterns that characterize the NAFTA countries from 2005-2009 may be too great, suppressing the impact of trade flows on public support because a lot can change over ten-plus years. Moreover, some of the agreements are implemented before 2005-09 and others afterwards.

To respond to these concerns, I have done three empirical things. First, note that the main text included a robustness check showing that the main results from Table \ref{tab4} (which uses data from 2005-09) are replicated using data from 2010-14. This only partly addresses the implementation question because all of these agreements were implemented by 2012. Second, to address the remaining questions, Table \ref{tabA5} employs new versions of the main independent variables, all measured two years after implementation of the agreement.\footnote{Note that the Middle East agreements have two agreements implemented in 2006 and one in 2009. I adopt 2006 as the year of implementation.} Note that doing so requires removing NAFTA from the data analysis, and so the predicted effects are smaller, in general, because firm support for NAFTA was so significant. This reduces the sample size by 403 industries. The model is otherwise identical to Table \ref{tab4} in the main text. Note that the direction and significance of the estimated effects are all very similar to \ref{tab4}. This suggests that the choice of trade patterns in 2005-09 is not unduly influencing the results. A complementary set of results is presented Table \ref{tabA6} which uses trade flows two years before the agreements were concluded. 

\setlength{\tabcolsep}{.3cm}
\begin{table}[t!]\centering
 \caption{Replication of Table \ref{tab4} with all trade flows measured two years after agreement implementation.} 
  \begin{threeparttable}
{\footnotesize \begin{tabular}{lD{.}{.}{2.5}D{.}{.}{2.5}D{.}{.}{2.5}D{.}{.}{2.5}}
\toprule
%\midrule
 & \multicolumn{2}{c}{\uline{$\ln$ variables}} & \multicolumn{2}{c}{\uline{rank \%-age vars.}} \vspace{3pt} \\
\multicolumn{1}{@{}l}{Outcome:} & \multicolumn{1}{c}{$\#$ Firms} & \multicolumn{1}{c}{Assoc.} & \multicolumn{1}{c}{$\#$ Firms} & \multicolumn{1}{c}{Assoc.}\\
% \cmidrule(l{1em}r{1em}){2}
\midrule
\multicolumn{5}{@{}l}{\uline{Related-party and intermediates trade}:} \vspace{2pt}\\
% latex table generated in R 3.0.3 by xtable 1.7-1 package
% Thu Apr 06 15:49:38 2017
 Rel. party imports & 0.32^{***} & 0.09^{***} & 0.53^{***} & 0.11^{***} \\ 
  Inputs & 0.57^{***} & 0.13^{***} & 1.51^{***} & 0.21^{***} \\ 
  Downstream exports & 0.04^{*} & 0.00 & -0.15^{*} & -0.02 \\ 
   \midrule \multicolumn{5}{@{}l}{\uline{Ordinary trade}:} \vspace{2pt}\\Imports $\times$ Homog. & 0.07 & -0.17^{***} & 0.05 & -0.11^{*} \\ 
  Imports $\times$ Diff. & 0.16 & 0.07^{*} & 0.39^{**} & 0.09^{**} \\ 
  Exports $\times$ Homog. & 0.23^{***} & 0.14^{***} & 0.48^{***} & 0.25^{***} \\ 
  Exports $\times$ Diff. & -0.17^{***} & -0.03^{*} & -0.76^{***} & -0.09^{***} \\ 
   \midrule \multicolumn{5}{@{}l}{\uline{Other controls}:} \vspace{2pt}\\Sales & 0.26^{***} & 0.01 & 1.21^{***} & 0.20^{***} \\ 
  Homog. $\rightarrow$ Mod. & -0.03 & -0.04 & 0.06 & -0.02 \\ 
  Homog. $\rightarrow$ Diff. & 0.25 & -0.14 & 0.45 & -0.10 \\ 
   \midrule  Pseudo-R$^2$ & 0.15 & 0.12 & 0.12 & 0.08 \\ 
  
Sample size & \multicolumn{1}{l}{\phantom{a}3627} & \multicolumn{1}{l}{\phantom{a}3627}  & \multicolumn{1}{l}{\phantom{a}3627} & \multicolumn{1}{l}{\phantom{a}3627}  \\
\bottomrule
\end{tabular}}
\begin{tablenotes}[para,flushleft]
\item
\leavevmode
  \kern-\scriptspace
  \kern-\labelsep
\scriptsize{\emph{Notes}:} {All estimates are first differences; changes in continuous variables are from 25$th$ to $75$th percentile. Median expected number of firms supporting is $1.13$; median expected probability of association support is $.29$. Standard errors are clustered at 3-digit NAICS-agreement level. \scriptsize \textsuperscript{***}$p<0.001$,\textsuperscript{**}$p<0.01$,\textsuperscript{*}$p<0.05$.}
\end{tablenotes}
  \end{threeparttable}
\label{tabA5}
\end{table}



\setlength{\tabcolsep}{.3cm}
\begin{table}[t!]\centering
 \caption{Replication of Table \ref{tab4} with all trade flows measured two years prior to implementation.} 
  \begin{threeparttable}
{\footnotesize \begin{tabular}{lD{.}{.}{2.5}D{.}{.}{2.5}D{.}{.}{2.5}D{.}{.}{2.5}}
\toprule
%\midrule
 & \multicolumn{2}{c}{\uline{$\ln$ variables}} & \multicolumn{2}{c}{\uline{rank \%-age vars.}} \vspace{3pt} \\
\multicolumn{1}{@{}l}{Outcome:} & \multicolumn{1}{c}{$\#$ Firms} & \multicolumn{1}{c}{Assoc.} & \multicolumn{1}{c}{$\#$ Firms} & \multicolumn{1}{c}{Assoc.}\\
% \cmidrule(l{1em}r{1em}){2}
\midrule
\multicolumn{5}{@{}l}{\uline{Related-party and intermediates trade}:} \vspace{2pt}\\
% latex table generated in R 3.0.3 by xtable 1.7-1 package
% Thu Apr 06 15:49:56 2017
 Rel. party imports & 0.33^{***} & 0.13^{***} & 0.49^{***} & 0.15^{***} \\ 
  Inputs & 0.29^{***} & 0.07^{***} & 0.05 & -0.05 \\ 
  Downstream exports & 0.10^{***} & 0.00 & 0.30^{**} & 0.05^{*} \\ 
   \midrule \multicolumn{5}{@{}l}{\uline{Ordinary trade}:} \vspace{2pt}\\Imports $\times$ Homog. & 0.04 & -0.22^{***} & 0.06 & -0.12^{*} \\ 
  Imports $\times$ Diff. & 0.13 & 0.03 & 0.28^{*} & 0.02 \\ 
  Exports $\times$ Homog. & 0.18^{**} & 0.08^{***} & 0.44^{**} & 0.15^{***} \\ 
  Exports $\times$ Diff. & 0.04 & 0.00 & -0.05 & 0.01 \\ 
   \midrule \multicolumn{5}{@{}l}{\uline{Other controls}:} \vspace{2pt}\\Sales & 0.37^{***} & 0.03^{**} & 0.84^{***} & 0.10^{***} \\ 
  Homog. $\rightarrow$ Mod. & -0.01 & -0.04 & 0.04 & -0.05 \\ 
  Homog. $\rightarrow$ Diff. & 0.40^{*} & -0.13 & 0.58 & -0.12^{*} \\ 
   \midrule  Pseudo-R$^2$ & 0.15 & 0.09 & 0.12 & 0.07 \\ 
  
Sample size & \multicolumn{1}{l}{\phantom{a}3324} & \multicolumn{1}{l}{\phantom{a}3224}  & \multicolumn{1}{l}{\phantom{a}3224} & \multicolumn{1}{l}{\phantom{a}3224}  \\
\bottomrule
\end{tabular}}
\begin{tablenotes}[para,flushleft]
\item
\leavevmode
  \kern-\scriptspace
  \kern-\labelsep
\scriptsize{\emph{Notes}:} {All estimates are first differences; changes in continuous variables are from 25$th$ to $75$th percentile. Median expected number of firms supporting is $1.13$; median expected probability of association support is $.29$. Standard errors are clustered at 3-digit NAICS-agreement level. \scriptsize \textsuperscript{***}$p<0.001$,\textsuperscript{**}$p<0.01$,\textsuperscript{*}$p<0.05$.}
\end{tablenotes}
  \end{threeparttable}
\label{tabA6}
\end{table}

As a final test, I have also reestimated the main models by excluding the NAFTA agreement (which is the most distant in time from 2005-09), and by excluding all of the agreements which entered into force before 2005. These results are presented in Table \ref{tabA7}. Columns 1 and 4 present the underlying regression models from Table \ref{tab4}'s columns 1 and 2. Columns 3 and 5 replicate the models among non-NAFTA observations; columns 4 and 6 present reestimate the models among agreements that entered into force in 2005 or after only. There are two main results here. First, the model fitting is not strongly driven by the NAFTA results in one direction or another. This is perhaps not surprising because NAFTA is only 1/10th of the data, however this is still reassuring. Second, the results with only 6 of the agreement clusters are substantively very similar. The effects of Related party imports and Downstream outputs are a little stronger and of imported inputs are weaker, but overall the main patterns uncovered still hold.

\setlength{\tabcolsep}{.12cm}
\begin{table}[t!]\centering
 \caption{Robustness of models 2 and 4 from Table \ref{tab4}.} 
  \begin{threeparttable}
{\footnotesize \begin{tabular}{lD{.}{.}{2.5}D{.}{.}{2.5}D{.}{.}{2.5}D{.}{.}{2.5}D{.}{.}{2.5}D{.}{.}{2.5}}
\toprule
& \multicolumn{3}{c}{\# firms} & \multicolumn{3}{c}{Assoc. support} \\ 
 & \multicolumn{1}{c}{1}  & \multicolumn{1}{c}{2} & \multicolumn{1}{c}{3} & \multicolumn{1}{c}{4} & \multicolumn{1}{c}{5} & \multicolumn{1}{c}{6} \\ 
\midrule
\multicolumn{5}{@{}l}{\uline{Related-party and intermediates trade}:} \vspace{2pt}\\
% latex table generated in R 3.0.3 by xtable 1.7-1 package
% Thu Apr 06 15:49:58 2017
 Rel. party imports &  0.031 &  0.032 &  0.039 &  0.035 &  0.042 &  0.057 \\ 
   & 0.006^{***} & 0.007^{***} & 0.008^{***} & 0.010^{***} & 0.010^{***} & 0.013^{***} \\ 
  Inputs (rel. party) &  0.232 &  0.186 &  0.088 &  0.193 &  0.209 &  0.147 \\ 
   & 0.012^{***} & 0.014^{***} & 0.021^{***} & 0.018^{***} & 0.022^{***} & 0.031^{***} \\ 
  Outputs &  0.027 &  0.015 &  0.016 &  0.004 & -0.004 &  0.005 \\ 
   & 0.005^{***} & 0.006^{*} & 0.007^{*} & 0.008 & 0.009 & 0.011 \\ 
   \midrule \multicolumn{5}{@{}l}{\uline{Ordinary trade}:} \vspace{2pt}\\Imports (non. rel. party) &  0.160 &  0.061 &  0.242 & -0.143 & -0.239 &  0.056 \\ 
   & 0.261 & 0.294 & 0.348 & 0.366 & 0.391 & 0.471 \\ 
  Exports &  0.862 &  0.906 &  1.128 &  0.270 &  0.127 &  0.505 \\ 
   & 0.253^{***} & 0.283^{**} & 0.336^{***} & 0.356 & 0.380 & 0.460 \\ 
  Imports $\times$ Mod. diff. &  0.020 &  0.009 &  0.004 &  0.075 &  0.079 &  0.062 \\ 
   & 0.016 & 0.017 & 0.020 & 0.022^{***} & 0.023^{***} & 0.027^{*} \\ 
  Imports $\times$ Diff. &  0.011 &  0.003 &  0.004 &  0.117 &  0.130 &  0.164 \\ 
   & 0.014 & 0.016 & 0.019 & 0.022^{***} & 0.023^{***} & 0.028^{***} \\ 
  Exports $\times$ Mod. diff. & -0.022 & -0.013 & -0.017 & -0.061 & -0.053 & -0.069 \\ 
   & 0.021 & 0.023 & 0.027 & 0.030^{*} & 0.032 & 0.038 \\ 
  Exports $\times$ Diff. & -0.040 & -0.044 & -0.051 & -0.162 & -0.158 & -0.221 \\ 
   & 0.020^{*} & 0.022^{*} & 0.025^{*} & 0.030^{***} & 0.031^{***} & 0.037^{***} \\ 
   \midrule \multicolumn{5}{@{}l}{\uline{Other controls}:} \vspace{2pt}\\Sales & -0.009 &  0.001 &  0.002 & -0.077 & -0.087 & -0.099 \\ 
   & 0.013 & 0.014 & 0.017 & 0.019^{***} & 0.020^{***} & 0.023^{***} \\ 
  Mod. diff &  0.040 &  0.045 &  0.044 &  0.125 &  0.126 &  0.154 \\ 
   & 0.018^{*} & 0.020^{*} & 0.022 & 0.026^{***} & 0.027^{***} & 0.032^{***} \\ 
  Diff. &  0.129 &  0.173 &  0.256 &  0.055 &  0.029 &  0.128 \\ 
   & 0.024^{***} & 0.028^{***} & 0.036^{***} & 0.037 & 0.041 & 0.052^{*} \\ 
  Intercept & -7.324 & -7.684 & -8.244 & -5.445 & -5.000 & -6.598 \\ 
   & 0.536^{***} & 0.614^{***} & 0.734^{***} & 0.807^{***} & 0.868^{***} & 1.048^{***} \\ 
  
\midrule
 N & \multicolumn{1}{c}{4030}  & \multicolumn{1}{c}{3627} & \multicolumn{1}{c}{2418} & \multicolumn{1}{c}{4030}  & \multicolumn{1}{c}{3627} & \multicolumn{1}{c}{2418} \\ 
\bottomrule
\end{tabular}}
\begin{tablenotes}[para,flushleft]
\item
\leavevmode
  \kern-\scriptspace
  \kern-\labelsep
\scriptsize{\emph{Notes}:} {\scriptsize \textsuperscript{***}$p<0.001$,\textsuperscript{**}$p<0.01$,\textsuperscript{*}$p<0.05$. Standard errors not clustered.}
\end{tablenotes}
  \end{threeparttable}
\label{tabA7}
\end{table}

\FloatBarrier
\newpage
\subsubsection*{Subset analysis}
The models from Table \ref{tab4} are reestimated using the terciles of the import-export ratio in Table \ref{tabA8}.

\setlength{\tabcolsep}{.12cm}
\begin{table}[t!]\centering
 \caption{Replication of Table \ref{tab4} in subsamples by export-import ratio.} 
  \begin{threeparttable}
{\footnotesize \begin{tabular}{lD{.}{.}{2.5}D{.}{.}{2.5}D{.}{.}{2.5}}
\toprule
\multicolumn{1}{@{}l}{Sample:} & \multicolumn{1}{c}{Net-importing} & \multicolumn{1}{c}{Balanced} & \multicolumn{1}{c}{Net-exporting}\\\midrule
\multicolumn{3}{@{}l}{\uline{Change in firm support, logged totals}:} \vspace{2pt}\\
% latex table generated in R 3.0.3 by xtable 1.7-1 package
% Thu Apr 06 15:50:04 2017
 Rel. party imports & 0.23^{**} & 0.28^{***} & 0.74^{***} \\ 
  Inputs & 1.01^{***} & 0.45^{***} & 1.05^{***} \\ 
  Downstream exports & 0.06 & -0.01 & 0.33^{***} \\ 
   \midrule \multicolumn{4}{@{}l}{\uline{Change in assoc. support, logged totals}:} \vspace{2pt}\\Rel. party imports & 0.13^{***} & 0.06^{***} & 0.06^{***} \\ 
  Inputs & 0.20^{***} & 0.10^{***} & 0.06^{**} \\ 
  Downstream exports & -0.01 & 0.01 & 0.02 \\ 
   \midrule \multicolumn{4}{@{}l}{\uline{Change in firm support, \%-age sales}:} \vspace{2pt}\\Rel. party imports & 0.30^{***} & 0.25^{***} & 1.23^{***} \\ 
  Inputs & 1.13^{***} & 0.46^{***} & 0.71^{***} \\ 
  Downstream exports & 0.08 & -0.03 & 0.70^{***} \\ 
   \midrule \multicolumn{4}{@{}l}{\uline{Change in assoc. support,  \%-age sales}:} \vspace{2pt}\\Rel. party imports & 0.14^{***} & 0.07^{***} & 0.09^{***} \\ 
  Inputs & 0.21^{***} & 0.10^{***} & 0.04 \\ 
  Downstream exports & -0.02 & 0.01 & 0.03 \\ 
  
\midrule
Sample size & \multicolumn{1}{l}{\phantom{aaa}1343} & \multicolumn{1}{l}{\phantom{a}1343} & \multicolumn{1}{l}{\phantom{aaa}1344} \\
\bottomrule
\end{tabular}}
\begin{tablenotes}[para,flushleft]
\item
\leavevmode
  \kern-\scriptspace
  \kern-\labelsep
\scriptsize{\emph{Notes}:} {All estimates are first differences; changes in continuous variables are from 25$th$ to $75$th percentile. \scriptsize \textsuperscript{***}$p<0.001$,\textsuperscript{**}$p<0.01$,\textsuperscript{*}$p<0.05$.}
\end{tablenotes}
  \end{threeparttable}
\label{tabA8}
\end{table}

\newpage
\FloatBarrier
\subsubsection*{Random and fixed effects models}
As a final set of checks, this paper examines a series of models with fixed effects and random effects. Random effects models introduce bias into coefficient estimates if covariates are correlated with the cluster intercepts but generally produce less variable estimates than fixed effects models. Rather than examine one side of this tradeoff, this paper considers both types of models across four potential groupings where separate intercepts might enter the regression function -- agreements; industries; 3-digit NAICS industries for each agreement; and, both agreements and industries at the same time.

These models are useful for demonstrating the variation off of which the estimates are generated. However, they may be misspecified for two related reasons. First, inter-country (or inter-industry) variation in the explanatory variables may be relevant in causing the outcomes of interest. To the extent that country- or industry-specific features are relevant in generating support for US trade agreements, they are likely captured in the rich covariates employed in each model. Second, fixed effects effectively demean the explanatory variables. It does not seem plausible that an industry with an average amount of exports to South Korea is just as motivated to support the KORUS as an industry with an average amount of exports to Jordan is motivated to support the US-Jordan FTA. To put it more starkly, imagine that there is a country to which the US exports nothing. A fixed effects model would predict that US firms are as supportive of a trade agreement with this autarkic country as with the US exporter to a NAFTA country, other than some idiosyncratic shock that just happened to cause greater support for NAFTA. That is not a plausible claim, given that the mean US industry exports over \$600 million annually to the NAFTA countries. For these reasons, I focus on these models to demonstrate only the relevant sources of variation.

Two required changes to the sample and models are made in order to facilitate the introduction of these intercepts. First, all industry-level variables are removed from the regression equation. Second, all observations where there is no variation in firm support (or association) support across an industry are also dropped from the models with industry-level intercepts. For the models incorporating industry-level fixed effects, this leaves 3730 for the firm support models and 2740 observations for the association support models. For the models incorporating, agreement-3 digit NAICS agreements, this leaves 3560 for the firm support models and 2859 observations for the association support models. Fixed effects also require a linear model, so I use the logged number of supporting firms and the association support dummy as outcome variables in a linear model set-up. I do the same for the random effects models; this is not necessary but makes the code much faster. An earlier version of the paper used random effects in negative binomial and logistic regression models, and found very similar results substantively.

Two observations on the results are critical. First, and as expected, the impact of related party imports, intermediate inputs, and downstream exports are all attenuated by the introduction of either random or fixed effects. Different agreements and different industries differ in their fixed characteristics, and these fixed characteristics, whatever they may be, are driving some of the position-taking. This does not mean that these differences are not themselves a consequence of our preferred explanatory factors -- there were more opportunities for multinationalization and inputs sourcing for all industries across NAFTA than the Jordan FTA, for example. We cannot be certain what drives these idiosyncratic agreement- or industry-specific factors. %, less of the variance may be explained by the factors emphasized here then appears at first blush. 

Second, the direction and statistical significance of the main findings in this paper are generally robust to the inclusion of random and fixed effects. This is particularly so in the case of the random effects, the agreement fixed effects, and for the firm support outcomes across the higher-dimensional industry fixed effects. Using the fixed effects (Tables A9-A12) sign changes are rare for the three main explanatory variables, and where they crop up are only statistically significant in one case involving association support with the Inputs variable and another involving firms support and the Downstream exports variable.  Using the random effects (Tables A13-A16) the only sign changes are for the association models with the proportion variables (model 2 of Table A16) where the Inputs variables is negative and significant and the Downstream exports variable is negative and not significant.  

\setlength{\tabcolsep}{.12cm}
\begin{table}[h!]\centering
\caption{Firm support for agreements with fixed effects.}   
 \begin{threeparttable}
{\footnotesize \begin{tabular}{lD{.}{.}{2.5}D{.}{.}{2.5}D{.}{.}{2.5}D{.}{.}{2.5}D{.}{.}{2.5}}
\toprule
% \midrule
\multicolumn{1}{@{}l}{DV: \# Firms} & \multicolumn{1}{c}{1}  & \multicolumn{1}{c}{2} & \multicolumn{1}{c}{3} & \multicolumn{1}{c}{4} & \multicolumn{1}{c}{5} \\ 
\midrule
\multicolumn{5}{@{}l}{\uline{Related-party and intermediates trade}:} \vspace{2pt}\\
% latex table generated in R 3.0.3 by xtable 1.7-1 package
% Thu Apr 06 15:50:06 2017
 Rel. party imports &  0.023 &  0.017 &  0.008 &  0.008 &  0.003 \\ 
   & 0.003^{***} & 0.003^{***} & 0.003^{**} & 0.002^{**} & 0.002 \\ 
  Inputs &  0.102 &  0.094 &  0.066 &  0.044 &  0.017 \\ 
   & 0.004^{***} & 0.006^{***} & 0.007^{***} & 0.005^{***} & 0.008^{*} \\ 
  Downstream exports &  0.010 &  0.007 &  0.015 &  0.153 &  0.052 \\ 
   & 0.002^{***} & 0.002^{**} & 0.002^{***} & 0.008^{***} & 0.009^{***} \\ 
   \midrule \multicolumn{5}{@{}l}{\uline{Ordinary trade}:} \vspace{2pt}\\Imports (non. rel. party) & -0.005 & -0.003 & -0.002 & -0.003 & -0.005 \\ 
   & 0.003 & 0.003 & 0.003 & 0.003 & 0.003 \\ 
  Exports &  0.010 &  0.010 &  0.005 &  0.022 & -0.001 \\ 
   & 0.002^{***} & 0.002^{***} & 0.002 & 0.005^{***} & 0.004 \\ 
   \midrule \multicolumn{5}{@{}l}{\uline{Other controls}:} \vspace{2pt}\\Intercept & -1.154 & -1.080 & -0.797 & -2.458 & -0.171 \\ 
   & 0.056^{***} & 0.085^{***} & 0.134^{***} & 0.183^{***} & 0.241 \\ 
  
\midrule
Agreement FE & \multicolumn{1}{c}{No} & \multicolumn{1}{c}{Yes} & \multicolumn{1}{c}{No} & \multicolumn{1}{c}{No} & \multicolumn{1}{c}{Yes} \\ 
  Agreement-NAICS3 FE & \multicolumn{1}{c}{No} & \multicolumn{1}{c}{No} & \multicolumn{1}{c}{Yes} & \multicolumn{1}{c}{No} & \multicolumn{1}{c}{No} \\ 
  6-digit NAICS FE & \multicolumn{1}{c}{No} & \multicolumn{1}{c}{No} & \multicolumn{1}{c}{No} & \multicolumn{1}{c}{Yes} & \multicolumn{1}{c}{Yes} \\ 
\midrule
 N & \multicolumn{1}{c}{4030}  & \multicolumn{1}{c}{4030} & \multicolumn{1}{c}{3560} & \multicolumn{1}{c}{3730} & \multicolumn{1}{c}{3730} \\ 
  R$^2$ & \multicolumn{1}{c}{0.32} & \multicolumn{1}{c}{0.40} & \multicolumn{1}{c}{0.58} & \multicolumn{1}{c}{0.62} & \multicolumn{1}{c}{0.69} \\ 
% \midrule 
\bottomrule
\end{tabular}}
\begin{tablenotes}[para,flushleft]
\item
\leavevmode
  \kern-\scriptspace
  \kern-\labelsep
\scriptsize{\emph{Notes}:} {\scriptsize \textsuperscript{***}$p<0.001$,\textsuperscript{**}$p<0.01$,\textsuperscript{*}$p<0.05$.}
\end{tablenotes}
  \end{threeparttable}
\label{tabA9}
\end{table}



\setlength{\tabcolsep}{.12cm}
\begin{table}[h!]\centering
\caption{Association support for agreements with fixed effects.}   
  \begin{threeparttable}
{\footnotesize \begin{tabular}{lD{.}{.}{2.5}D{.}{.}{2.5}D{.}{.}{2.5}D{.}{.}{2.5}D{.}{.}{2.5}}
\toprule
% \midrule
\multicolumn{1}{@{}l}{DV: Assoc. support} & \multicolumn{1}{c}{1}  & \multicolumn{1}{c}{2} & \multicolumn{1}{c}{3} & \multicolumn{1}{c}{4} & \multicolumn{1}{c}{5} \\ 
\midrule
\multicolumn{5}{@{}l}{\uline{Related-party and intermediates trade}:} \vspace{2pt}\\
% latex table generated in R 3.0.3 by xtable 1.7-1 package
% Thu Apr 06 15:50:07 2017
 Rel. party imports &  0.008 &  0.007 &  0.005 &  0.004 &  0.005 \\ 
   & 0.002^{***} & 0.002^{***} & 0.002 & 0.002 & 0.002^{*} \\ 
  Inputs &  0.034 &  0.014 &  0.043 &  0.035 &  0.010 \\ 
   & 0.003^{***} & 0.004^{***} & 0.006^{***} & 0.005^{***} & 0.007 \\ 
  Downstream exports &  0.004 &  0.003 &  0.007 &  0.047 &  0.032 \\ 
   & 0.002^{*} & 0.001 & 0.002^{***} & 0.008^{***} & 0.009^{***} \\ 
   \midrule \multicolumn{5}{@{}l}{\uline{Ordinary trade}:} \vspace{2pt}\\Imports (non. rel. party) & -0.002 & -0.005 & -0.007 &  0.005 &  0.002 \\ 
   & 0.002 & 0.002^{*} & 0.003^{**} & 0.003 & 0.003 \\ 
  Exports &  0.006 &  0.008 &  0.008 &  0.011 &  0.010 \\ 
   & 0.002^{***} & 0.002^{***} & 0.002^{***} & 0.004^{**} & 0.004^{**} \\ 
   \midrule \multicolumn{5}{@{}l}{\uline{Other controls}:} \vspace{2pt}\\Intercept & -0.303 & -0.028 & -0.363 & -1.491 & -0.837 \\ 
   & 0.038^{***} & 0.061 & 0.113^{**} & 0.147^{***} & 0.216^{***} \\ 
  
\midrule
Agreement FE & \multicolumn{1}{c}{No} & \multicolumn{1}{c}{Yes} & \multicolumn{1}{c}{No} & \multicolumn{1}{c}{No} & \multicolumn{1}{c}{Yes} \\ 
  Agreement-NAICS3 FE & \multicolumn{1}{c}{No} & \multicolumn{1}{c}{No} & \multicolumn{1}{c}{Yes} & \multicolumn{1}{c}{No} & \multicolumn{1}{c}{No} \\ 
  6-digit NAICS FE & \multicolumn{1}{c}{No} & \multicolumn{1}{c}{No} & \multicolumn{1}{c}{No} & \multicolumn{1}{c}{Yes} & \multicolumn{1}{c}{Yes} \\ 
\midrule
N & \multicolumn{1}{c}{4030}  & \multicolumn{1}{c}{4030} & \multicolumn{1}{c}{2859} & \multicolumn{1}{c}{2740} & \multicolumn{1}{c}{2740} \\ 
  R$^2$ & \multicolumn{1}{c}{0.11} & \multicolumn{1}{c}{0.15} & \multicolumn{1}{c}{0.32} & \multicolumn{1}{c}{0.42} & \multicolumn{1}{c}{0.50} \\ 
% \midrule 
\bottomrule
\end{tabular}}
\begin{tablenotes}[para,flushleft]
\item
\leavevmode
  \kern-\scriptspace
  \kern-\labelsep
\scriptsize{\emph{Notes}:}{\scriptsize \textsuperscript{***}$p<0.001$,\textsuperscript{**}$p<0.01$,\textsuperscript{*}$p<0.05$.}
\end{tablenotes}
  \end{threeparttable}
\label{tabA10}
\end{table}




\setlength{\tabcolsep}{.12cm}
\begin{table}[h!]\centering
\caption{Firm support for agreements with fixed effects.}   
  \begin{threeparttable}
{\footnotesize \begin{tabular}{lD{.}{.}{2.5}D{.}{.}{2.5}D{.}{.}{2.5}D{.}{.}{2.5}D{.}{.}{2.5}}
\toprule
% \midrule
\multicolumn{1}{@{}l}{DV: \# Firms} & \multicolumn{1}{c}{1}  & \multicolumn{1}{c}{2} & \multicolumn{1}{c}{3} & \multicolumn{1}{c}{4} & \multicolumn{1}{c}{5} \\ 
\midrule
\multicolumn{5}{@{}l}{\uline{Related-party and intermediates trade}:} \vspace{2pt}\\
% latex table generated in R 3.0.3 by xtable 1.7-1 package
% Thu Apr 06 15:50:08 2017
 Rel. party imports &  0.156 &  0.151 &  0.066 &  0.020 &  0.007 \\ 
   & 0.019^{***} & 0.019^{***} & 0.018^{***} & 0.018 & 0.016 \\ 
  Inputs &  0.175 &  0.065 & -0.016 &  0.111 &  0.041 \\ 
   & 0.011^{***} & 0.015^{***} & 0.018 & 0.012^{***} & 0.014^{**} \\ 
  Downstream exports &  0.023 & -0.025 &  0.035 &  0.265 &  0.053 \\ 
   & 0.010^{*} & 0.010^{*} & 0.011^{***} & 0.020^{***} & 0.021^{*} \\ 
   \midrule \multicolumn{5}{@{}l}{\uline{Other controls}:} \vspace{2pt}\\Imports (non. rel. party) & -0.077 & -0.099 & -0.047 &  0.009 & -0.025 \\ 
   & 0.019^{***} & 0.018^{***} & 0.018^{**} & 0.019 & 0.018 \\ 
  Exports &  0.143 &  0.113 &  0.060 &  0.174 &  0.055 \\ 
   & 0.013^{***} & 0.013^{***} & 0.013^{***} & 0.017^{***} & 0.016^{***} \\ 
   \midrule \multicolumn{5}{@{}l}{\uline{Ordinary trade}:} \vspace{2pt}\\Intercept & -0.218 &  0.110 &  0.097 & -0.617 &  0.493 \\ 
   & 0.027^{***} & 0.053^{*} & 0.112 & 0.169^{***} & 0.170^{**} \\ 
  
\midrule
Agreement FE & \multicolumn{1}{c}{No} & \multicolumn{1}{c}{Yes} & \multicolumn{1}{c}{No} & \multicolumn{1}{c}{No} & \multicolumn{1}{c}{Yes} \\ 
  Agreement-NAICS3 FE & \multicolumn{1}{c}{No} & \multicolumn{1}{c}{No} & \multicolumn{1}{c}{Yes} & \multicolumn{1}{c}{No} & \multicolumn{1}{c}{No} \\ 
  6-digit NAICS FE & \multicolumn{1}{c}{No} & \multicolumn{1}{c}{No} & \multicolumn{1}{c}{No} & \multicolumn{1}{c}{Yes} & \multicolumn{1}{c}{Yes} \\ 
\midrule
N & \multicolumn{1}{c}{4030}  & \multicolumn{1}{c}{4030} & \multicolumn{1}{c}{3560} & \multicolumn{1}{c}{3730} & \multicolumn{1}{c}{3730} \\ 
  R$^2$ & \multicolumn{1}{c}{0.26} & \multicolumn{1}{c}{0.34} & \multicolumn{1}{c}{0.56} & \multicolumn{1}{c}{0.61} & \multicolumn{1}{c}{0.69} \\ 
% \midrule 
\bottomrule
\end{tabular}}
\begin{tablenotes}[para,flushleft]
\item
\leavevmode
  \kern-\scriptspace
  \kern-\labelsep
\scriptsize{\emph{Notes}:} {\scriptsize \textsuperscript{***}$p<0.001$,\textsuperscript{**}$p<0.01$,\textsuperscript{*}$p<0.05$.}
\end{tablenotes}
  \end{threeparttable}
\label{tabA11}
\end{table}


\setlength{\tabcolsep}{.12cm}
\begin{table}[h!]\centering
\caption{Association support for agreements with fixed effects.}   
  \begin{threeparttable}
{\footnotesize \begin{tabular}{lD{.}{.}{2.5}D{.}{.}{2.5}D{.}{.}{2.5}D{.}{.}{2.5}D{.}{.}{2.5}}
\toprule
% \midrule
\multicolumn{1}{@{}l}{DV: Assoc. support} & \multicolumn{1}{c}{1}  & \multicolumn{1}{c}{2} & \multicolumn{1}{c}{3} & \multicolumn{1}{c}{4} & \multicolumn{1}{c}{5} \\ 
\midrule
\multicolumn{5}{@{}l}{\uline{Related-party and intermediates trade}:} \vspace{2pt}\\
% latex table generated in R 3.0.3 by xtable 1.7-1 package
% Thu Apr 06 15:50:09 2017
 Rel. party imports &  0.042 &  0.053 &  0.045 &  0.008 &  0.019 \\ 
   & 0.013^{**} & 0.013^{***} & 0.016^{**} & 0.012 & 0.012 \\ 
  Inputs &  0.055 & -0.027 & -0.021 &  0.056 &  0.017 \\ 
   & 0.007^{***} & 0.010^{**} & 0.017 & 0.008^{***} & 0.010 \\ 
  Downstream exports &  0.009 & -0.010 &  0.020 &  0.061 &  0.041 \\ 
   & 0.007 & 0.007 & 0.009^{*} & 0.014^{***} & 0.015^{**} \\ 
   \midrule \multicolumn{5}{@{}l}{\uline{Other controls}:} \vspace{2pt}\\Imports (non. rel. party) & -0.010 & -0.034 & -0.062 &  0.022 & -0.001 \\ 
   & 0.013 & 0.013^{**} & 0.016^{***} & 0.013 & 0.013 \\ 
  Exports &  0.054 &  0.049 &  0.054 &  0.061 &  0.047 \\ 
   & 0.009^{***} & 0.009^{***} & 0.011^{***} & 0.012^{***} & 0.012^{***} \\ 
   \midrule \multicolumn{5}{@{}l}{\uline{Ordinary trade}:} \vspace{2pt}\\Intercept &  0.017 &  0.249 &  0.232 & -0.511 & -0.331 \\ 
   & 0.018 & 0.037^{***} & 0.090^{*} & 0.115^{***} & 0.123^{**} \\ 
  
\midrule
Agreement FE & \multicolumn{1}{c}{No} & \multicolumn{1}{c}{Yes} & \multicolumn{1}{c}{No} & \multicolumn{1}{c}{No} & \multicolumn{1}{c}{Yes} \\ 
  Agreement-NAICS3 FE & \multicolumn{1}{c}{No} & \multicolumn{1}{c}{No} & \multicolumn{1}{c}{Yes} & \multicolumn{1}{c}{No} & \multicolumn{1}{c}{No} \\ 
  6-digit NAICS FE & \multicolumn{1}{c}{No} & \multicolumn{1}{c}{No} & \multicolumn{1}{c}{No} & \multicolumn{1}{c}{Yes} & \multicolumn{1}{c}{Yes} \\ 
\midrule
N & \multicolumn{1}{c}{4030}  & \multicolumn{1}{c}{4030} & \multicolumn{1}{c}{2859} & \multicolumn{1}{c}{2740} & \multicolumn{1}{c}{2740} \\ 
  R$^2$ & \multicolumn{1}{c}{0.09} & \multicolumn{1}{c}{0.15} & \multicolumn{1}{c}{0.31} & \multicolumn{1}{c}{0.50} & \multicolumn{1}{c}{0.54} \\ 
% \midrule 
\bottomrule
\end{tabular}}
\begin{tablenotes}[para,flushleft]
\item
\leavevmode
  \kern-\scriptspace
  \kern-\labelsep
\scriptsize{\emph{Notes}:} {\scriptsize \textsuperscript{***}$p<0.001$,\textsuperscript{**}$p<0.01$,\textsuperscript{*}$p<0.05$.}
\end{tablenotes}
  \end{threeparttable}
\label{tabA12}
\end{table}



\setlength{\tabcolsep}{.12cm}
\begin{table}[h!]\centering
 \caption{Firm support for agreements with random intercepts.} 
  \begin{threeparttable}
{\footnotesize \begin{tabular}{lD{.}{.}{2.5}D{.}{.}{2.5}D{.}{.}{2.5}D{.}{.}{2.5}D{.}{.}{2.5}}
\toprule
% \midrule
\multicolumn{1}{@{}l}{DV: \# Firms} & \multicolumn{1}{c}{1}  & \multicolumn{1}{c}{2} & \multicolumn{1}{c}{3} & \multicolumn{1}{c}{4} & \multicolumn{1}{c}{5} \\ 
\midrule
\multicolumn{5}{@{}l}{\uline{Related-party and intermediates trade}:} \vspace{2pt}\\
% latex table generated in R 3.0.3 by xtable 1.7-1 package
% Thu Apr 06 15:50:10 2017
 Rel. party imports &  0.023 &  0.017 &  0.010 &  0.015 &  0.006 \\ 
   & 0.003^{***} & 0.003^{***} & 0.003^{***} & 0.002^{***} & 0.002^{*} \\ 
  Inputs &  0.102 &  0.095 &  0.080 &  0.089 &  0.039 \\ 
   & 0.004^{***} & 0.006^{***} & 0.006^{***} & 0.005^{***} & 0.007^{***} \\ 
  Downstream exports &  0.010 &  0.007 &  0.015 &  0.045 &  0.017 \\ 
   & 0.002^{***} & 0.002^{**} & 0.002^{***} & 0.004^{***} & 0.004^{***} \\ 
   \midrule \multicolumn{5}{@{}l}{\uline{Ordinary trade}:} \vspace{2pt}\\Imports (non. rel. party) & -0.005 & -0.003 & -0.001 & -0.004 & -0.004 \\ 
   & 0.003 & 0.003 & 0.003 & 0.003 & 0.003 \\ 
  Exports &  0.010 &  0.010 &  0.004 &  0.024 &  0.010 \\ 
   & 0.002^{***} & 0.002^{***} & 0.002 & 0.004^{***} & 0.003^{**} \\ 
   \midrule \multicolumn{5}{@{}l}{\uline{Other controls}:} \vspace{2pt}\\Intercept & -1.154 & -0.971 & -0.711 & -1.530 & -0.209 \\ 
   & 0.056^{***} & 0.104^{***} & 0.096^{***} & 0.066^{***} & 0.161 \\ 
  
\midrule
Agreement RE & \multicolumn{1}{c}{No} & \multicolumn{1}{c}{Yes} & \multicolumn{1}{c}{No} & \multicolumn{1}{c}{No} & \multicolumn{1}{c}{Yes} \\ 
  Agreement-NAICS3 RE & \multicolumn{1}{c}{No} & \multicolumn{1}{c}{No} & \multicolumn{1}{c}{Yes} & \multicolumn{1}{c}{No} & \multicolumn{1}{c}{No} \\ 
  6-digit NAICS RE & \multicolumn{1}{c}{No} & \multicolumn{1}{c}{No} & \multicolumn{1}{c}{No} & \multicolumn{1}{c}{Yes} & \multicolumn{1}{c}{Yes} \\ 
\midrule
N & \multicolumn{1}{c}{4030}  & \multicolumn{1}{c}{4030} & \multicolumn{1}{c}{3560} & \multicolumn{1}{c}{3730} & \multicolumn{1}{c}{3730} \\ 
% \midrule 
\bottomrule
\end{tabular}}
\begin{tablenotes}[para,flushleft]
\item
\leavevmode
  \kern-\scriptspace
  \kern-\labelsep
\scriptsize{\emph{Notes}:} {\scriptsize \textsuperscript{***}$p<0.001$,\textsuperscript{**}$p<0.01$,\textsuperscript{*}$p<0.05$.}
\end{tablenotes}
  \end{threeparttable}
\label{tab13}
\end{table}



\setlength{\tabcolsep}{.12cm}
\begin{table}[t!]\centering
 \caption{Association support for agreements with random intercepts.} 
  \begin{threeparttable}
{\footnotesize \begin{tabular}{lD{.}{.}{2.5}D{.}{.}{2.5}D{.}{.}{2.5}D{.}{.}{2.5}D{.}{.}{2.5}}
\toprule
% \midrule
\multicolumn{1}{@{}l}{DV: Assoc. support} & \multicolumn{1}{c}{1}  & \multicolumn{1}{c}{2} & \multicolumn{1}{c}{3} & \multicolumn{1}{c}{4} & \multicolumn{1}{c}{5} \\ 
\midrule
\multicolumn{5}{@{}l}{\uline{Related-party and intermediates trade}:} \vspace{2pt}\\
% latex table generated in R 3.0.3 by xtable 1.7-1 package
% Thu Apr 06 15:50:11 2017
 Rel. party imports &  0.008 &  0.007 &  0.005 &  0.007 &  0.006 \\ 
   & 0.002^{***} & 0.002^{***} & 0.002^{*} & 0.002^{**} & 0.002^{**} \\ 
  Inputs &  0.034 &  0.016 &  0.037 &  0.052 &  0.013 \\ 
   & 0.003^{***} & 0.004^{***} & 0.005^{***} & 0.004^{***} & 0.006^{*} \\ 
  Downstream exports &  0.004 &  0.003 &  0.006 &  0.008 &  0.001 \\ 
   & 0.002^{*} & 0.001 & 0.002^{***} & 0.003^{*} & 0.003 \\ 
   \midrule \multicolumn{5}{@{}l}{\uline{Ordinary trade}:} \vspace{2pt}\\Imports (non. rel. party) & -0.002 & -0.005 & -0.006 &  0.002 & -0.001 \\ 
   & 0.002 & 0.002^{*} & 0.003^{*} & 0.003 & 0.002 \\ 
  Exports &  0.006 &  0.008 &  0.007 &  0.009 &  0.008 \\ 
   & 0.002^{***} & 0.002^{***} & 0.002^{**} & 0.003^{***} & 0.002^{***} \\ 
   \midrule \multicolumn{5}{@{}l}{\uline{Other controls}:} \vspace{2pt}\\Intercept & -0.303 & -0.040 & -0.207 & -0.543 &  0.120 \\ 
   & 0.038^{***} & 0.067 & 0.079^{**} & 0.053^{***} & 0.109 \\ 
  
\midrule
Agreement RE & \multicolumn{1}{c}{No} & \multicolumn{1}{c}{Yes} & \multicolumn{1}{c}{No} & \multicolumn{1}{c}{No} & \multicolumn{1}{c}{Yes} \\ 
  Agreement-NAICS3 RE & \multicolumn{1}{c}{No} & \multicolumn{1}{c}{No} & \multicolumn{1}{c}{Yes} & \multicolumn{1}{c}{No} & \multicolumn{1}{c}{No} \\ 
  6-digit NAICS RE & \multicolumn{1}{c}{No} & \multicolumn{1}{c}{No} & \multicolumn{1}{c}{No} & \multicolumn{1}{c}{Yes} & \multicolumn{1}{c}{Yes} \\ 
\midrule
 N & \multicolumn{1}{c}{4030}  & \multicolumn{1}{c}{4030} & \multicolumn{1}{c}{2859} & \multicolumn{1}{c}{2740} & \multicolumn{1}{c}{2740} \\ 
% \midrule 
\bottomrule
\end{tabular}}
\begin{tablenotes}[para,flushleft]
\item
\leavevmode
  \kern-\scriptspace
  \kern-\labelsep
\scriptsize{\emph{Notes}:} {\scriptsize \textsuperscript{***}$p<0.001$,\textsuperscript{**}$p<0.01$,\textsuperscript{*}$p<0.05$.}
\end{tablenotes}
  \end{threeparttable}
\label{tabA14}
\end{table}


\setlength{\tabcolsep}{.12cm}
\begin{table}[t!]\centering
 \caption{Firm support for agreements with random intercepts.}  
  \begin{threeparttable}
{\footnotesize \begin{tabular}{lD{.}{.}{2.5}D{.}{.}{2.5}D{.}{.}{2.5}D{.}{.}{2.5}D{.}{.}{2.5}}
\toprule
% \midrule
\multicolumn{1}{@{}l}{DV: \# Firms} & \multicolumn{1}{c}{1}  & \multicolumn{1}{c}{2} & \multicolumn{1}{c}{3} & \multicolumn{1}{c}{4} & \multicolumn{1}{c}{5} \\ 
\midrule
\multicolumn{5}{@{}l}{\uline{Related-party and intermediates trade}:} \vspace{2pt}\\
% latex table generated in R 3.0.3 by xtable 1.7-1 package
% Thu Apr 06 15:50:11 2017
 Rel. party imports &  0.505 &  0.466 &  0.208 &  0.149 &  0.063 \\ 
   & 0.084^{***} & 0.080^{***} & 0.080^{**} & 0.085 & 0.079 \\ 
  Inputs &  0.558 &  0.098 &  0.043 &  0.472 &  0.086 \\ 
   & 0.048^{***} & 0.063 & 0.074 & 0.053^{***} & 0.066 \\ 
  Downstream exports &  0.211 &  0.006 &  0.152 &  0.646 &  0.145 \\ 
   & 0.043^{***} & 0.042 & 0.046^{***} & 0.068^{***} & 0.068^{*} \\ 
   \midrule \multicolumn{5}{@{}l}{\uline{Ordinary trade}:} \vspace{2pt}\\Imports (non. rel. party) & -0.161 & -0.259 & -0.081 &  0.052 & -0.064 \\ 
   & 0.083 & 0.079^{**} & 0.080 & 0.090 & 0.084 \\ 
  Exports &  0.471 &  0.353 &  0.202 &  0.676 &  0.314 \\ 
   & 0.057^{***} & 0.054^{***} & 0.057^{***} & 0.071^{***} & 0.068^{***} \\ 
   \midrule \multicolumn{5}{@{}l}{\uline{Other controls}:} \vspace{2pt}\\Intercept & -1.468 &  0.386 &  0.934 & -2.207 &  0.751 \\ 
   & 0.116^{***} & 0.468 & 0.260^{***} & 0.152^{***} & 0.555 \\ 
  
\midrule
Agreement RE & \multicolumn{1}{c}{No} & \multicolumn{1}{c}{Yes} & \multicolumn{1}{c}{No} & \multicolumn{1}{c}{No} & \multicolumn{1}{c}{Yes} \\ 
  Agreement-NAICS3 RE & \multicolumn{1}{c}{No} & \multicolumn{1}{c}{No} & \multicolumn{1}{c}{Yes} & \multicolumn{1}{c}{No} & \multicolumn{1}{c}{No} \\ 
  6-digit NAICS RE & \multicolumn{1}{c}{No} & \multicolumn{1}{c}{No} & \multicolumn{1}{c}{No} & \multicolumn{1}{c}{Yes} & \multicolumn{1}{c}{Yes} \\ 
\midrule
N & \multicolumn{1}{c}{4030}  & \multicolumn{1}{c}{4030} & \multicolumn{1}{c}{3560} & \multicolumn{1}{c}{3730} & \multicolumn{1}{c}{3730} \\ 
% \midrule 
\bottomrule
\end{tabular}}
\begin{tablenotes}[para,flushleft]
\item
\leavevmode
  \kern-\scriptspace
  \kern-\labelsep
\scriptsize{\emph{Notes}:} {\scriptsize \textsuperscript{***}$p<0.001$,\textsuperscript{**}$p<0.01$,\textsuperscript{*}$p<0.05$.}
\end{tablenotes}
  \end{threeparttable}
\label{tabA15}
\end{table}



\setlength{\tabcolsep}{.12cm}
\begin{table}[t!]\centering
 \caption{Association support for agreements with random intercepts.}  
  \begin{threeparttable}
{\footnotesize \begin{tabular}{lD{.}{.}{2.5}D{.}{.}{2.5}D{.}{.}{2.5}D{.}{.}{2.5}D{.}{.}{2.5}}
\toprule
% \midrule
\multicolumn{1}{@{}l}{DV: Assoc. support} & \multicolumn{1}{c}{1}  & \multicolumn{1}{c}{2} & \multicolumn{1}{c}{3} & \multicolumn{1}{c}{4} & \multicolumn{1}{c}{5} \\ 
\midrule
\multicolumn{5}{@{}l}{\uline{Related-party and intermediates trade}:} \vspace{2pt}\\
% latex table generated in R 3.0.3 by xtable 1.7-1 package
% Thu Apr 06 15:50:24 2017
 Rel. party imports &  0.187 &  0.282 &  0.270 &  0.112 &  0.201 \\ 
   & 0.066^{**} & 0.070^{***} & 0.086^{**} & 0.093 & 0.101^{*} \\ 
  Inputs &  0.276 & -0.118 &  0.021 &  0.628 &  0.136 \\ 
   & 0.038^{***} & 0.054^{*} & 0.072 & 0.060^{***} & 0.081 \\ 
  Downstream exports &  0.037 & -0.049 &  0.099 &  0.252 & -0.016 \\ 
   & 0.033 & 0.035 & 0.047^{*} & 0.075^{***} & 0.088 \\ 
   \midrule \multicolumn{5}{@{}l}{\uline{Ordinary trade}:} \vspace{2pt}\\Imports (non. rel. party) & -0.048 & -0.192 & -0.278 &  0.168 &  0.051 \\ 
   & 0.066 & 0.070^{**} & 0.087^{**} & 0.096 & 0.105 \\ 
  Exports &  0.264 &  0.250 &  0.277 &  0.466 &  0.341 \\ 
   & 0.045^{***} & 0.046^{***} & 0.060^{***} & 0.074^{***} & 0.083^{***} \\ 
   \midrule \multicolumn{5}{@{}l}{\uline{Other controls}:} \vspace{2pt}\\Intercept & -2.276 & -1.281 & -0.989 & -3.456 & -1.671 \\ 
   & 0.099^{***} & 0.324^{***} & 0.228^{***} & 0.187^{***} & 0.530^{**} \\ 
  
\midrule
Agreement RE & \multicolumn{1}{c}{No} & \multicolumn{1}{c}{Yes} & \multicolumn{1}{c}{No} & \multicolumn{1}{c}{No} & \multicolumn{1}{c}{Yes} \\ 
  Agreement-NAICS3 RE & \multicolumn{1}{c}{No} & \multicolumn{1}{c}{No} & \multicolumn{1}{c}{Yes} & \multicolumn{1}{c}{No} & \multicolumn{1}{c}{No} \\ 
  6-digit NAICS RE & \multicolumn{1}{c}{No} & \multicolumn{1}{c}{No} & \multicolumn{1}{c}{No} & \multicolumn{1}{c}{Yes} & \multicolumn{1}{c}{Yes} \\ 
\midrule
N & \multicolumn{1}{c}{4030}  & \multicolumn{1}{c}{4030} & \multicolumn{1}{c}{2859} & \multicolumn{1}{c}{2740} & \multicolumn{1}{c}{2740} \\ 
% \midrule 
\bottomrule
\end{tabular}}
\begin{tablenotes}[para,flushleft]
\item
\leavevmode
  \kern-\scriptspace
  \kern-\labelsep
\scriptsize{\emph{Notes}:} {\scriptsize \textsuperscript{***}$p<0.001$,\textsuperscript{**}$p<0.01$,\textsuperscript{*}$p<0.05$.}
\end{tablenotes}
  \end{threeparttable}
\label{tabA16}
\end{table}

\end{document}


\FloatBarrier

\newpage \footnotesize
\setcounter{table}{0}
\renewcommand{\thetable}{B\arabic{table}}
\setcounter{figure}{0}
\renewcommand{\thefigure}{B\arabic{figure}}

\newpage
\section*{Appendix B: Cases and Data}

\subsubsection*{Cases}
This paper considers data from the following trade agreements and other reciprocal liberalization measures. 


\begin{quotation}
\noindent Australia-US Free Trade Agreement (AUSFTA) \\
Dominican Republic-Central America Free Trade Agreement (CAFTA-DR) \\
Chile Free Trade Agreement \\
Free Trade Agreement of the Americas (Failed) \\
Jordan Free Trade Agreement \\
Korea-US Free Trade Agreement (KORUS) \\
U.S.-Panama Trade Agreement and U.S.-Colombia Trade Agreement (Treated jointly) \\
Bahrain Free Trade Agreement; Morocco Free Trade Agreement; Oman Free Trade Agreement; United Arab Emirates-US Free Trade Agreement (Not ratified) (Treated jointly) \\
Peru-US Trade Promotion Agreement \\
Singapore-US Free Trade Agreement \\
Permanent Normal Trade Relations with China \\
Permanent Normal Trade Relations with Russia 
\end{quotation}

\subsubsection*{Imputed positions}
A small number of association positions are imputed, because the agreement explicitly excluded liberalization of a particular industry, whether their own or a supplying upstream industry. All of these cases are listed below, and the reasoning is mentioned, in brief. These cases are all associations, because they tend to occur in agricultural industries. ``Oppose:ImputeFavor" implies that the association publicly opposed the agreement, but they are treated as likely supporter, should the agreement have included full liberalization.

\begin{quotation}

\noindent \textbf{Australia-US Free Trade Agreement (AUSFTA)} \\
National Association of Wheat Growers;	Oppose:ImputeFavor \\
U.S. Wheat Associates;	Oppose:ImputeFavor \\
Wheat Export Trade Education Committee; Oppose:ImputeFavor\\
-Australia was permitted to maintain its Wheat Board, which certain agricultural groups contended would suppress competition in the Australian market.\\
American Sugar Alliance;  Favor:ImputeOppose\\
Grocery Manufacturers Association; Oppose:ImputeFavor\\
-The US Sugar industry avoided substantive liberalization; "sugar was excluded from the agreement..."\footnote{USITC. "US-Australia Free Trade Agreement: Potential Economywide and Selected Sectoral Effects." pg. xv.} The GMA opposed the agreement because sugar was excluded, and beef, to an extent. The agreement "...includes no increases for imports of Australian-grown sugar and only minimal increases for beef and dairy products."\footnote{Congressional Research Service. "Agriculture in the Australia-U.S. Free Trade Agreement". September 29 2004. http://www.cnie.org/nle/crsreports/briefingbooks/Agriculture/Agriculture\%20in\%20the\%20Australia-US.htm. Accessed on: October 19 2014.}
\bigskip

\noindent \textbf{Korea-US Free Trade Agreement (KORUS)} \\
USA Rice Federation;	Oppose:ImputeFavor \\
US Rice Producers Association;	Oppose:ImputeFavor \\
Rice Millers' Association;	Oppose:ImputeFavor \\
-The rice producing associations opposed the agreement because rice was largely excluded from the agreement. E.g. ``USA Rice does not support the agreement as it stands due to the exclusion of rice. Free trade agreements entered into by the US should be comprehensive and include all products even those that are politically sensistive."\footnote{U.S. Congress. House of Representatives. Committee on Ways and Means. 2011. The Pending Free Trade Agreements with Colombia, Panama and South Korea and the Creation of US Jobs. 112th Congress, 1st session, January 25. http://www.gpo.gov/fdsys/pkg/CHRG-112hhrg67469/html/CHRG-112hhrg67469.htm Accessed on: October 2, 2014.}
\bigskip


\noindent \textbf{US-Peru Trade Promotion Agreement} \\
Travel Goods Association; Oppose:ImputeFavor \\
-"The Travel Goods Association (TGA) does not support the U.S.-Peru TPA. TGA states that the TPA has highly restrictive provisions on textile travel goods that
prevent U.S. travel goods companies from using the best available inputs."\footnote{United States International Trade Commission. 2006. U.S.-Peru Trade Promotion Agreement: Potential Economy-wide and Selected
Sectoral Effects. Investigation No. TA-2104-20. From USITC Website. http://www.usitc.gov/publications/docs/pubs/2104f/pub3855.pdf Accessed on October 7 2014. pg. 3-25.}
\end{quotation}


\subsubsection*{Additional explanatory variables}

\textbf{Proxies for FDI}: As an alternative to the data on related-party imports as a proxy for vertical FDI, this paper also examines a measure constructed using data on US direct investment abroad (DIA) provided by the Bureau of Economic Analysis. The measure of current US direct investment abroad in the trade partner is used to reweight the measure of US direct investment abroad in the entire world. Formally, the measure is given by:
\begin{defi}
The alternative proxy for US multinational activity abroad is measured at the four-digit NAICS level (indexed by $j$) but employs data on US Direct Investment Abroad at the two-digit NAICS level (indexed by $i$ where $j \in i$):
$$ \mathrm{DIA}_j = \mathrm{DIA}^{US,\,Partner}_i\cdot \frac{\mathrm{DIA}^{US}_j}{\sum_{j \in i} \mathrm{DIA}^{US}_j}. $$
\end{defi} 
\noindent We refer to this variable as DIA in Table \ref{tab3}. In the sample of trade agreements examined here, the Spearman correlation between the DIA variables and the related party imports measure is $.423$. The Pearson correlation of the logged variables is $.439$.

\begin{table}[t!!]\centering\footnotesize
  \begin{threeparttable}
\setlength{\tabcolsep}{.15cm}
\caption{US Direct Investment in treaty partners and their FDI in the US}
\begin{tabular}{lrrrrrr}
\midrule
& \multicolumn{3}{@{}c}{\uline{All sectors}} & \multicolumn{3}{@{}c}{\uline{Manufacturing only}} \vspace{3pt}\\
\multicolumn{1}{@{}l}{Agreement} & US DIA & FDI in US & DIA/FDI &  US DIA & FDI in US & DIA/FDI \\
\midrule
NAFTA & 388200 & 211300 & 1.8 & 104800 & 42200 & 2.5 \\ 
  Jordan & 100 & 0 & $ \infty $ & 0 & 0 & - \\ 
  Australia & 120000 & 42500 & 2.8 & 14500 & 6300 & 2.3 \\ 
  Chile & 22800 & 400 & 57.0 & 3200 & 0 & $ \infty $ \\ 
  Singapore & 113000 & 16700 & 6.8 & 17500 & 7700 & 2.3 \\ 
  CAFTA & 4900 & 0 & $ \infty $ & 2200 & 0 & $ \infty $ \\ 
  Bah/Mor/Oman & 200 & 200 & 1.0 & 100 & 0 & $ \infty $ \\ 
  Peru & 6000 & 100 & 60.0 & 700 & 0 & $ \infty $ \\ 
  Col/Pan & 11100 & 3700 & 3.0 & 1800 & 100 & 18.0 \\ 
  Korea & 39500 & 18200 & 2.2 & 11700 & 2400 & 4.9 \\ 
\midrule
World & 3632600 & 2255800 & 1.6 & 526400 & 743400 & 0.7 \\  
   \bottomrule
\end{tabular}
\begin{tablenotes}[para,flushleft]
\item
\leavevmode
  \kern-\scriptspace
  \kern-\labelsep
\scriptsize{\emph{Notes}:} {All FDI and DIA figures are Position on a historical cost basis in millions of dollars, and averaged over 2005-2014.}
\end{tablenotes}
  \end{threeparttable}
\label{invest}
\end{table}

I also provide in Table \ref{invest} information on relative direct investment abroad by US firms, and foreign direct investment in the US by foreign firms, using data from the BEA. These figures are provided for all country partners. The main point of this table is that US investment in the agreement partner countries has tended to vastly exceed investment by those same countries in the United States. This suggests that Related-party imports are a reliable proxy for imports to the US by American multinational firms, because such imports are unlikely to be coming from foreign firms invested in the United States.


\small
\clearpage
\bibliography{WAIDbib}



\end{document}